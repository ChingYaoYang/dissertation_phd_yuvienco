% -*-latex-*-
% 
% For questions, comments, concerns or complaints:
% thesis@mit.edu
% 
%
% $Log: cover.tex,v $
% Revision 1.8  2008/05/13 15:02:15  jdreed
% Degree month is June, not May.  Added note about prevdegrees.
% Arthur Smith's title updated
%
% Revision 1.7  2001/02/08 18:53:16  boojum
% changed some \newpages to \cleardoublepages
%
% Revision 1.6  1999/10/21 14:49:31  boojum
% changed comment referring to documentstyle
%
% Revision 1.5  1999/10/21 14:39:04  boojum
% *** empty log message ***
%
% Revision 1.4  1997/04/18  17:54:10  othomas
% added page numbers on abstract and cover, and made 1 abstract
% page the default rather than 2.  (anne hunter tells me this
% is the new institute standard.)
%
% Revision 1.4  1997/04/18  17:54:10  othomas
% added page numbers on abstract and cover, and made 1 abstract
% page the default rather than 2.  (anne hunter tells me this
% is the new institute standard.)
%
% Revision 1.3  93/05/17  17:06:29  starflt
% Added acknowledgements section (suggested by tompalka)
% 
% Revision 1.2  92/04/22  13:13:13  epeisach
% Fixes for 1991 course 6 requirements
% Phrase "and to grant others the right to do so" has been added to 
% permission clause
% Second copy of abstract is not counted as separate pages so numbering works
% out
% 
% Revision 1.1  92/04/22  13:08:20  epeisach

% NOTE:
% These templates make an effort to conform to the MIT Thesis specifications,
% however the specifications can change.  We recommend that you verify the
% layout of your title page with your thesis advisor and/or the MIT 
% Libraries before printing your final copy.
\pagenumbering{roman}
\title{Modulation of the Physicochemical Properties of Block Protein Copolymers
    via Fluorination
and the Adaptation of Coiled-coil Proteins for Small Molecule Delivery Applications}

\author{Carlo Yuvienco}
% If you wish to list your previous degrees on the cover page, use the 
% previous degrees command:
%       \prevdegrees{A.A., Harvard University (1985)}
% You can use the \\ command to list multiple previous degrees
%       \prevdegrees{B.S., University of California (1978) \\
%                    S.M., Massachusetts Institute of Technology (1981)}
\department{Department of Chemical and Biomolecular Engineering}

% If the thesis is for two degrees simultaneously, list them both
% separated by \and like this:
% \degree{Doctor of Philosophy \and Master of Science}
\degree{Doctor of Philosophy}
\degreetitle{Biomedical Engineering}
\idnumber{N13876012}

% As of the 2007-08 academic year, valid degree months are September, 
% February, or June.  The default is June.
\degreemonth{September}
\degreeyear{2014}
\thesisdate{September 19, 2014}

%% By default, the thesis will be copyrighted to MIT.  If you need to copyright
%% the thesis to yourself, just specify the `vi' documentclass option.  If for
%% some reason you want to exactly specify the copyright notice text, you can
%% use the \copyrightnoticetext command.  
%\copyrightnoticetext{\copyright IBM, 1990.  Do not open till Xmas.}

% If there is more than one supervisor, use the \supervisor command
% once for each.
\advisor{Prof. Jin Kim Montclare, Ph.D.}

% This is the department committee chairman, not the thesis committee
% chairman.  You should replace this with your Department's Committee
% Chairman.
\chairman{David Pine}{Chairman, Department Head}

% Make the titlepage based on the above information.  If you need
% something special and can't use the standard form, you can specify
% the exact text of the titlepage yourself.  Put it in a titlepage
% environment and leave blank lines where you want vertical space.
% The spaces will be adjusted to fill the entire page.  The dotted
% lines for the signatures are made with the \signature command.
\maketitle

\cleardoublepage

% -*- Mode:TeX -*-
%
% Some departments (e.g. Chemistry) require an additional cover page
% with signatures of the thesis committee.  Please check with your
% thesis advisor or other appropriate person to determine if such a 
% page is required for your thesis.  
%
% If you choose not to use the "titlepage" environment, a \newpage
% commands, and several \vspace{\fill} commands may be necessary to
% achieve the required spacing.  The \signature command is defined in
% the "mitthesis" class
%
% The following sample appears courtesy of Ben Kaduk <kaduk@mit.edu> and
% was used in his June 2012 doctoral thesis in Chemistry. 

\begin{titlepage}
\raggedright
Approved by the Guidance Committee:

\signatureentry{Professor Jin Ryoun Kim}{Associate Professor of Chemical Engineering}

\signatureentry{Professor Jin Kim Montclare}{Associate Professor of Chemistry}

\signatureentry{Professor Rastislav Levicky}{Associate Professor of Chemical Engineering}

\signatureentry{Professor Stacy Blain}{Associate Professor of Molecular Biology}

\signatureentry{Professor Raymond Tu}{Associate Professor of Chemical Engineering}


\end{titlepage}



\cleardoublepage

% --------------------------
\singlespacing
\noindent
Microfilm or copies of this dissertation may be obtained from:\\
\\
UMI Dissertation Publishing\\
ProQuest USA\\
789 E. Eisenhower Parkway\\
P.O. Box 1346\\
Ann Arbor, MI 48106-1346
% --------------------------

\cleardoublepage

\doublespacing
\section*{Vita}
Carlo Yuvienco was born in New York, New York on March 22\textsuperscript{nd},
1983 to Dr. Francisco P. Yuvienco and Elizabeth T. Yuvienco. His academic and
professional pursuits would reside in New York until the time of this work,
starting with his secondary education at Stuyvesant High School. This was
followed by an undergraduate education in chemical engineering at the Cooper
Union for the Advancement of Science and Art, culminating in a Bachelor of
Engineering (granted in 2005). Carlo then was employed at Pall Corporation (East
Hills, NY) as a research scientist, working on the development of
leukocyte-reduction filtration media. After two years of industry experience,
Carlo pursued a Master of Science in biomedical engineering at the Polytechnic
Institute of New York University (granted in 2010), immediately preceding the
pursuit of his doctorate in biomedical engineering under the advisement of Prof.
Jin Kim Montclare, Ph.D. at the now New York University - Polytechnic School of
Engineering, in the Department of Chemical and Biomolecular Engineering. From
2008 to 2014, Carlo worked on several projects in the field of protein
engineering under the guidance of Prof. Montclare, having conducted experiments
in various labs within New York University, the Navy Research Laboratory, the
Wright-Patterson Air Force Base Materials and Manufacturing Directorate, and the
City University of New York. During the course of his doctoral research, he and
his research projects have been supported by the National Science Foundation
(GK-12 Program, DGE-0741714; NYU MRSEC Center, DMR-0820341; I-Corps Program,
IIP-1332165; DMR-1205384), the Air Force Office of Scientific Research
(FA-9550-07-1-0060 and FA-9550-08-1-0266), and the Army Research Office
(W911NP-10-1-0228).


\cleardoublepage

\section*{Acknowledgments}
This body of work encompasses seven years of effort that would not have been
possible without the professional and emotional support of my friends, family,
and newfound colleagues. I declare my immense thanks to them for the
following reasons.

First, to my family:\\
The continuous pride and understanding that my family has expressed in regards
to the pursuit of my doctorate has been the driving force behind my stalwart
efforts these past few years. They have continued to champion my marathon
despite my long-awaited presence at the finish line. And during the times of despair
and frustration that commonly plague scientific research, I have always been
able to benefit from the respite that I received in the sanctuaries that were
their homes. To all of them, I convey my eternal gratitude and love.

To my colleague, friend, and mentor - Jin Montclare:\\
From day one, you have continued to foster and promote my advancement as a
scientist and intellectual. Assuredly, I am a better person for it.  We have
celebrated academic successes as well as suffered scientific and professional
tribulations. But perhaps most importantly, regardless of the context, you have
undoubtedly and consistently demonstrated your utmost trust in both my abilities
and my character - a privilege for which I can only hope I have abundantly
reciprocated in kind. Thank you very much for the momentous opportunity to have
worked with you.

To my colleagues:\\
There are many individuals to whom I owe thanks for their professional support
during the course of my doctoral studies. They include fellow students (high
school, undergraduate, and graduate alike), accomplished scientists, aspiring
professors, and university administrators and staff. A vast majority of them
have provided me with an infinitely expansive sounding board for my ideas and
have in turn stimulated the advancement of my research. Others have quelled
quagmires before they escalated toward affecting the completion of my degree.
And others have simply promoted my advancement as a research scientist through
the extension of rare professional opportunities. 

To my close friends:\\
I submit my sincerest apologies, more so and in addition to my gratitude. A
graduate career of seven years has in truth robbed me of many occasions to spend
with my close personal friends. They have nevertheless continued to support
my pursuit and take pride in my endeavors.

And to my Jennifer:\\
Without an iota of doubt, my professional accomplishments would not
have been possible without the love and support that you have devoted to me. You
have kept me company during late night experiments. You have consoled me during
times of struggle and sadness. You have cheered my successes and announced them
atop mountains. You are my rock and my cushion. And in my seven-year pursuit of
knowledge of the unknown, I have come to only one surety - that I love you.

To all of you again - thank you.


% The abstractpage environment sets up everything on the page except
% the text itself.  The title and other header material are put at the
% top of the page, and the supervisors are listed at the bottom.  A
% new page is begun both before and after.  Of course, an abstract may
% be more than one page itself.  If you need more control over the
% format of the page, you can use the abstract environment, which puts
% the word "Abstract" at the beginning and single spaces its text.

%% You can either \input (*not* \include) your abstract file, or you can put
%% the text of the abstract directly between the \begin{abstractpage} and
%% \end{abstractpage} commands.

% First copy: start a new page, and save the page number.
\cleardoublepage
% Uncomment the next line if you do NOT want a page number on your
% abstract and acknowledgments pages.
% \pagestyle{empty}
%\setcounter{savepage}{\thepage}
\begin{abstractpage}
% $Log: abstract.tex,v $
% Revision 1.1  93/05/14  14:56:25  starflt
% Initial revision
% 
% Revision 1.1  90/05/04  10:41:01  lwvanels
% Initial revision
% 
%
%% The text of your abstract and nothing else (other than comments) goes here.
%% It will be single-spaced and the rest of the text that is supposed to go on
%% the abstract page will be generated by the abstractpage environment.  This
%% file should be \input (not \include 'd) from cover.tex.
In this thesis, I designed and implemented a compiler which performs
optimizations that reduce the number of low-level floating point operations
necessary for a specific task; this involves the optimization of chains of
floating point operations as well as the implementation of a ``fixed'' point
data type that allows some floating point operations to simulated with integer
arithmetic.  The source language of the compiler is a subset of C, and the
destination language is assembly language for a micro-floating point CPU.  An
instruction-level simulator of the CPU was written to allow testing of the
code.  A series of test pieces of codes was compiled, both with and without
optimization, to determine how effective these optimizations were.

\end{abstractpage}

% Additional copy: start a new page, and reset the page number.  This way,
% the second copy of the abstract is not counted as separate pages.
% Uncomment the next 6 lines if you need two copies of the abstract
% page.
% \setcounter{page}{\thesavepage}
 %\begin{abstractpage}
 %% $Log: abstract.tex,v $
% Revision 1.1  93/05/14  14:56:25  starflt
% Initial revision
% 
% Revision 1.1  90/05/04  10:41:01  lwvanels
% Initial revision
% 
%
%% The text of your abstract and nothing else (other than comments) goes here.
%% It will be single-spaced and the rest of the text that is supposed to go on
%% the abstract page will be generated by the abstractpage environment.  This
%% file should be \input (not \include 'd) from cover.tex.
In this thesis, I designed and implemented a compiler which performs
optimizations that reduce the number of low-level floating point operations
necessary for a specific task; this involves the optimization of chains of
floating point operations as well as the implementation of a ``fixed'' point
data type that allows some floating point operations to simulated with integer
arithmetic.  The source language of the compiler is a subset of C, and the
destination language is assembly language for a micro-floating point CPU.  An
instruction-level simulator of the CPU was written to allow testing of the
code.  A series of test pieces of codes was compiled, both with and without
optimization, to determine how effective these optimizations were.

 %\end{abstractpage}

%%%%%%%%%%%%%%%%%%%%%%%%%%%%%%%%%%%%%%%%%%%%%%%%%%%%%%%%%%%%%%%%%%%%%%
% -*-latex-*-
