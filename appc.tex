\chapter{Supplementary data}

\section{Solid-state \NMR*{19,F} on \emph{p}FF-CE\textsubscript{3} protein}

All experiments were carried out using a Varian 500 MHz spectrometer operating
at \SI{470.6}{\mega\hertz} for \textsuperscript{19}F.\footnote{\NMR*{19,F}
experiments were carried out with the generous assistance and supervision of Dr.
Christopher Klug (Naval Research Laboratory, Chemistry Division).}
% --------------------------
The sample was packed into a \SI{3.2}{\mm} SiN rotor and spun at
\SI{20}{\kilo\hertz}.  The mass of the sample was \SI{10.5}{\mg}.  All data was
acquired using a rotor synchronized spin echo, with varying amounts of 1H
decoupling. The measurements were run at two temperatures, \SI{10}{\celsius} and
\SI{35}{\celsius}.

Typical spectra are shown in Figure \ref{fig:nmr_result_1}.  The narrow peak at
roughly \SI{-122}{\ppm} is assigned to residual Teflon present in the Torlon
components of the NMR rotor.  The relatively broad peak at ~\SI{115}{\ppm}
associated with \emph{p}FF-CE\textsubscript{3} reflects a distribution of local
environments for the \textsuperscript{19}F nuclei of these side groups.  (Note
that the other peaks in the spectra are spinning sidebands.) There is no
significant difference in the results obtained at the two temperatures.
% --------------------------
\begin{figure}[h!] \centering \includegraphics[width=0.8\textwidth]{f_c_05}
    \caption{\NMR*{19,F} spectra obtained at \SI{10}{\celsius} and
    \SI{35}{\celsius} for \emph{p}FF-CE\textsubscript{3} protein.}
    \label{fig:nmr_result_1} \end{figure}
% --------------------------

As the decoupling power increases there is first a minimum peak intensity for
the \emph{p}FF-CE\textsubscript{3} sample at 1000 followed by a gradual
increase (see Figure \ref{fig:nmr_result_2}).  As expected, the Teflon peak was
relatively insensitive to \textsuperscript{1}H decoupling.  Higher decoupling
powers were not attempted to avoid excessive sample heating.

% --------------------------
\begin{figure}
    \centering
    \begin{subfigure}[b]{0.4\textwidth}
        \includegraphics[width=\textwidth]{f_c_06a}
    \end{subfigure}
    \begin{subfigure}[b]{0.4\textwidth}
        \includegraphics[width=\textwidth]{f_c_06b}
    \end{subfigure}
    \caption{\NMR*{19,F} peak intensities for the residual Teflon (green) and
    \emph{p}FF-CE\textsubscript{3} (red) on the \textsuperscript{1}H decoupling
    power parameter aHtppm at (a) \SI{10}{\celsius} and (b)
    \SI{35}{\celsius}.  Spectra suggest a dependence of the peak intensities for
    the residual Teflon (green) and \emph{p}FF-CE\textsubscript{3} (red) on the
    \textsuperscript{1}H decoupling power parameter \emph{aHtppm} at
    \SI{10}{\celsius} and \SI{35}{\celsius}.}\label{fig:nmr_result_2}
\end{figure}
% --------------------------

Figure \ref{fig:nmr_result_3} shows the dependence of the peak intensities in
the \NMR*{19,F} spectra on time between spin echoes, d1.  This gives a measure
of the spin-lattice relaxation time, T1.  The solid lines are fits to a simple
single-exponential recovery corresponding to a single T1.  While there is a
slight decrease in the calculated T1 for the Teflon from ~2.1 s to 1.8 s as the
temperature increased from \SI{10}{\celsius} to \SI{35}{\celsius}.  There is no
clear change in the calculated T1 for \emph{p}FF-CE\textsubscript{3}, which
remains at ~2.2-2.3 s.  This somewhat unexpected result may be due to a broad
distribution in relaxation behavior for the 19F in
\emph{p}FF-CE\textsubscript{3}.
% --------------------------
\begin{figure}
    \centering
    \begin{subfigure}[b]{0.4\textwidth}
        \includegraphics[width=\textwidth]{f_c_07a}
    \end{subfigure}
    \begin{subfigure}[b]{0.4\textwidth}
        \includegraphics[width=\textwidth]{f_c_07b}
    \end{subfigure}
    \caption{\NMR*{19,F} peak intensities for the (a) residual Teflon and (b)
    \emph{p}FF-CE\textsubscript{3} as a function of the time between spin
    echoes, d1, at \SI{10}{\celsius} (green) and \SI{35}{\celsius} (red).}
    \label{fig:nmr_result_3}
\end{figure}
% --------------------------


\section{The cytotoxic effects of protein constructs}
\begin{figure}[h!] \centering \includegraphics[width=0.95\textwidth]{f_c_01}
    \caption{The effects of COMP and block copolymer protein constructs on the
        viability of MC3T3-E1 preosteoblasts, assessed via the MTT assay method,
        \emph{n}=4. wt-CE, \emph{p}FF-CE, and COMP were prepared to a final
        concentration of \SI{200}{\ug\per\mL} and \emph{p}FF was prepared to a
        final concentration of \SI{0.8}{\ug\per\mL}.}\label{fig:mtt_results}
    \end{figure}
Cytotoxicity of the protein constructs were evaluated by
3-(4,5-dimethylthiazol-2-yl)-2,5-diphenyl tetrasodium bromide, MTT assay.
MC3T3-E1 preosteoblast cells (
\SI[scientific-notation=true,retain-unity-mantissa=true]{1e6}{\per\ml}) in
\SI{100}{\uL} of ${\alpha}$MEM (Invitrogen) supplemented with \SI{10}{\volper}
FBS were seeded in 96-well plates and incubated overnight. The
\SI{5}{\mg\per\mL} MTT reagent in PBS was added into the plates and incubated
for 4 h. After incubation, the medium was aspirated and dimethyl sulfoxide
(\SI{100}{\uL} per well) was added to stop the reaction. The optical density was
then quantified in a microplate reader, Synergy HT at \SI{570}{\nm} wavelength.
The percentage of cell viability was calculated by comparing the groups to
control cells, which did not contain any of the sample reagents. The results are
shown in Figure \ref{fig:mtt_results}. the majority of the constructs
demonstrate a statistically insignificant effect on the viability of cells as
compared to the control.


\section{Transmission electron microscopy study of COMP coiled-coil protein}

\begin{figure}
    \centering
    \begin{subfigure}[b]{0.31\textwidth}
        \includegraphics[width=\textwidth]{f_c_02a}
        \caption{}
    \end{subfigure}
    \begin{subfigure}[b]{0.31\textwidth}
        \includegraphics[width=\textwidth]{f_c_02b}
        \caption{}
    \end{subfigure}
    \begin{subfigure}[b]{0.31\textwidth}
        \includegraphics[width=\textwidth]{f_c_02c}
        \caption{}
    \end{subfigure}
    \begin{subfigure}[b]{0.31\textwidth}
        \includegraphics[width=\textwidth]{f_c_02d}
        \caption{}
    \end{subfigure}
    \begin{subfigure}[b]{0.31\textwidth}
        \includegraphics[width=\textwidth]{f_c_02e}
        \caption{}
    \end{subfigure}
    \begin{subfigure}[b]{0.31\textwidth}
        \includegraphics[width=\textwidth]{f_c_02f}
        \caption{}
    \end{subfigure}
    \begin{subfigure}[b]{0.31\textwidth}
        \includegraphics[width=\textwidth]{f_c_02g}
        \caption{}
    \end{subfigure}
    \begin{subfigure}[b]{0.31\textwidth}
        \includegraphics[width=\textwidth]{f_c_02h}
        \caption{}
    \end{subfigure}
    \begin{subfigure}[b]{0.31\textwidth}
        \includegraphics[width=\textwidth]{f_c_02i}
        \caption{}
    \end{subfigure}
    \caption{Electron micrographs of COMP protein prepared to }\label{fig:COMP_EM_3}
\end{figure}

\section{Limulus amebocyte lysate (LAL) assay of COMP protein preparation}
\label{sec:lal_assay}
\begin{figure}[h!] \centering \includegraphics[width=0.7\textwidth]{f_c_03}
    \caption{Quantitation of endotoxin levels in COMP protein preparations
        following endotoxin removal using a Detoxi-Gel column. Open square
        markers denote standard curve data points, derived from an
        \latin{E.coli} O111:B4 standard. Closed markers indicate COMP elution
        samples from the column separation. Samples were applied to the assay
        kit as a 1:120, 1:25, and 1:10 dilution samples for pre-endotoxin
        separation, \SI{300}{\milli\moLar}, and \SI{500}{\milli\moLar} elutions,
    respectively. Correlation coefficient of 0.92881 was calculated from a
linear regression fit along the standard data points.}\label{fig:LAL_assay}
\end{figure}

\section{Size-exclusion standard curves}

\begin{figure}
    \centering
    \begin{subfigure}[b]{0.8\textwidth}
        \includegraphics[width=\textwidth]{f_c_04a}
        \caption{}
    \end{subfigure}
    \begin{subfigure}[b]{0.8\textwidth}
        \includegraphics[width=\textwidth]{f_c_04b}
        \caption{}
    \end{subfigure}
    \caption{Size-exclusion chromatograms collected for lysozyme and bovine
        serum albumin, each run as separate \SI{2}{\micro\liter} injections and
        prepared to \SI{3}{\mg\per\mL} in \SI{100}{\milli\moLar} Gomori
        phosphate buffer, pH 8.0, \SI{10}{\volper} methanol, consistent with
        mobile phase conditions. (a) Overlay of chromatograms of protein
        standards in the absence of BMS493 in the mobile phase buffer. (b)
        Overlay of chromatograms of protein standards in the presence of BMS493
        in the mobile phase buffer.}\label{fig:sec_standards}
\end{figure}

\section{UPLC method development}
\label{sec:uplc_method}
% --------------------------
\begin{figure}[h!] \centering \begin{subfigure}[b]{0.7\textwidth}
        \includegraphics[width=\textwidth]{f_2_13a} \caption{UPLC gradient for
        chromatographic separation} \label{fig:uplc_gradient} \end{subfigure}
    \begin{subfigure}[b]{0.7\textwidth}
        \includegraphics[width=\textwidth]{f_2_13b} \caption{Separation of BMS493, ATRA, and retinyl palmitate, collected at
            \SI{329}{\nm}, with noted retention times of \SI{2.176}{\minute},
            \SI{2.372}{\minute}, and \SI{7.151}{\minute}, respectively.}
        \label{fig:std_chromatogram} \end{subfigure}
    \begin{subfigure}[b]{0.7\textwidth}
        \includegraphics[width=\textwidth]{f_2_13c} \caption{Cross-sectional
            spectra
            at points of peak maxima, corresponding to ATRA, BMS493, and retinyl
        palmitate (top to bottom).}
        \label{fig:retinoid_spectra} \end{subfigure}
    \caption[Chromatographic separation of BMS493, ATRA, and retinyl
    palmitate]{Chromatographic separation of BMS493, ATRA, and retinyl palmitate
        using the UPLC gradient shown in (a). The resultant chromatogram (b)
        demonstrate successful separation of this control sample. From 3D
        spectra (data not shown), single retention time peak spectra were
    extracted for the identification of optimal wavelengths for
analytes.}\label{fig:uplc_report} \end{figure}
% --------------------------
UPLC method was optimized by first attempting separation of BMS493, retinoic
acid, and retinyl palmitate using a binary system of acetonitrile and water.
However, this resulted in poor resolution of retinoic acid. The addition of
methanol was then added, in agreement with previously reported
methods.\cite{DeLeenheer1982,Kane2008b,Wang2001a,Schaffer2010} However, BMS493
and retinoic were shown to co-elute easily with each other. To optimize the
resolution of BMS493 and endogenous retinoids, but also to maintain maximum
attainable peak heights (and signal/noise), a linear scouting gradient was
performed from 10:10:80 (ACN:MetOH:\ch{H2O}, \SI{0.1}{\percent} formic acid) to
45:45:10 (ACN:MetOH:\ch{H2O}, \SI{0.1}{\percent} formic acid) over
\SI{30}{\minute} (data not shown). From this optimization, maximal resolution
and minimum peak width were determined to be obtained using an isocratic mobile
phase consisting of 41.2:41.2:17.6 (ACN:MetOH:\ch{H2O}, \SI{0.1}{\percent}
formic acid). Elution of retinyl palmitate requires a relatively abrupt gradient
to \SI{100}{\percent} ACN, then held isocratically for approximately
\SI{3}{\minute} (Figure \ref{fig:uplc_gradient}). The cross-sectional spectra
presented in Figure \ref{fig:retinoid_spectra} allowed for single channels -
\SI{355}{\nm}, \SI{329}{\nm}, and \SI{324}{\nm} for ATRA, BMS493, and retinyl
palmitate, respectively - to be collected to obtain the best sensitivity of
detection as well as cross-channel calibration against retinyl palmitate.

% --------------------------
\begin{figure}[h!]
    \centering
    \begin{subfigure}[b]{0.75\textwidth}
        \includegraphics[width=\textwidth]{f_2_14a}
        \caption{MilliQ, \SI{0.3366}{\ng}, \SI{92.46}{\percent} recovery}
        \label{fig:milliq_extract}
    \end{subfigure}
    \begin{subfigure}[b]{0.8\textwidth}
        \includegraphics[width=\textwidth]{f_2_14b}
        \caption{DMEM, \SI{0.335}{\ng}, \SI{92.02}{\percent} recovery}
        \label{fig:lonza_extract}
    \end{subfigure} \caption{UPLC quantitation of BMS493 (\SI{3.1}{\minute}) in
    extract controls prepared in different media. Each sample was spiked with
    \SI{0.1}{\volper} \SI{1}{\milli\moLar} BMS493 prepared in
DMSO}\label{fig:extract_controls} \end{figure}
% --------------------------
