\chapter{Supplementary data}

\section{The cytotoxic effects of protein constructs}
\begin{figure}[h!] \centering \includegraphics[width=0.95\textwidth]{f_c_01}
    \caption{The effects of COMP and block copolymer protein constructs on the
        viability of MC3T3-E1 preosteoblasts, assessed via the MTT assay method,
        \emph{n}=4. wt-CE, \emph{p}FF-CE, and COMP were prepared to a final
        concentration of \SI{200}{\ug\per\mL} and \emph{p}FF was prepared to a
        final concentration of \SI{0.8}{\ug\per\mL}.}\label{fig:mtt_results}
    \end{figure}
Cytotoxicity of the protein constructs were evaluated by
3-(4,5-dimethylthiazol-2-yl)-2,5-diphenyl tetrasodium bromide, MTT assay.
MC3T3-E1 proosteoblast cells (
\SI[scientific-notation=true,retain-unity-mantissa=true]{1e6}{\per\ml}) in
\SI{100}{\uL} of ${\alpha}$MEM (Invitrogen) supplemented with \SI{10}{\volper}
FBS were seeded in 96-well plates and incubated overnight. The
\SI{5}{\mg\per\mL} MTT reagent in PBS was added into the plates and incubated
for 4 h. After incubation, the medium was aspirated and dimethyl sulfoxide
(\SI{100}{\uL} per well) was added to stop the reaction. The optical density was
then quantified in a microplate reader, Synergy HT at \SI{570}{\nm} wavelength.
The percentage of cell viability was calculated by comparing the groups to
control cells, which did not contain any of the sample reagents. The results are
shown in Figure \ref{fig:mtt_results}. the majority of the constructs
demonstrate a statistically insignificant effect on the viability of cells as
compared to the control.


\section{Transmission electron microscopy study of COMP coiled-coil protein}

\begin{figure}
    \centering
    \begin{subfigure}[b]{0.31\textwidth}
        \includegraphics[width=\textwidth]{f_c_02a}
        \caption{}
    \end{subfigure}
    \begin{subfigure}[b]{0.31\textwidth}
        \includegraphics[width=\textwidth]{f_c_02b}
        \caption{}
    \end{subfigure}
    \begin{subfigure}[b]{0.31\textwidth}
        \includegraphics[width=\textwidth]{f_c_02c}
        \caption{}
    \end{subfigure}
    \begin{subfigure}[b]{0.31\textwidth}
        \includegraphics[width=\textwidth]{f_c_02d}
        \caption{}
    \end{subfigure}
    \begin{subfigure}[b]{0.31\textwidth}
        \includegraphics[width=\textwidth]{f_c_02e}
        \caption{}
    \end{subfigure}
    \begin{subfigure}[b]{0.31\textwidth}
        \includegraphics[width=\textwidth]{f_c_02f}
        \caption{}
    \end{subfigure}
    \begin{subfigure}[b]{0.31\textwidth}
        \includegraphics[width=\textwidth]{f_c_02g}
        \caption{}
    \end{subfigure}
    \begin{subfigure}[b]{0.31\textwidth}
        \includegraphics[width=\textwidth]{f_c_02h}
        \caption{}
    \end{subfigure}
    \begin{subfigure}[b]{0.31\textwidth}
        \includegraphics[width=\textwidth]{f_c_02i}
        \caption{}
    \end{subfigure}
    \caption{Electron micrographs of COMP protein prepared to }\label{fig:COMP_EM_3}
\end{figure}

\section{Limulus amebocyte lysate (LAL) assay of COMP protein preparation}
\label{sec:lal_assay}
\begin{figure}[h!] \centering \includegraphics[width=0.7\textwidth]{f_c_03}
    \caption{Quantitation of endotoxin levels in COMP protein preparations
        following endotoxin removal using a Detoxi-Gel column. Open square
        markers denote standard curve data points, derived from an
        \latin{E.coli} O111:B4 standard. Closed markers indicate COMP elution
        samples from the column separation. Samples were applied to the assay
        kit as a 1:120, 1:25, and 1:10 dilution samples for pre-endotoxin
        separation, \SI{300}{\milli\moLar}, and \SI{500}{\milli\moLar} elutions,
    respectively. Correlation coefficient of 0.92881 was calculated from a
linear regression fit along the standard data points.}\label{fig:LAL_assay}
\end{figure}

\section{Size-exclusion standard curves}

\begin{figure}
    \centering
    \begin{subfigure}[b]{0.8\textwidth}
        \includegraphics[width=\textwidth]{f_c_04a}
        \caption{}
    \end{subfigure}
    \begin{subfigure}[b]{0.8\textwidth}
        \includegraphics[width=\textwidth]{f_c_04b}
        \caption{}
    \end{subfigure}
    \caption{Size-exclusion chromatograms collected for lysozyme and bovine
        serum albumin, each run as separate \SI{2}{\micro\liter} injections and
        prepared to \SI{3}{\mg\per\mL} in \SI{100}{\milli\moLar} Gomori
        phosphate buffer, pH 8.0, \SI{10}{\volper} methanol, consistent with
        mobile phase conditions. (a) Overlay of chromatograms of protein
        standards in the absence of BMS493 in the mobile phase buffer. (b)
        Overlay of chromatograms of protein standards in the presence of BMS493
        in the mobile phase buffer.}\label{fig:sec_standards}
\end{figure}
