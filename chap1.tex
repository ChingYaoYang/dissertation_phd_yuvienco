\chapter{Modulation of Assembly Properties of Protein-based Biomaterials via
Global Fluorinated Amino Acid Incorporation}
\label{chap:pff}

\begin{refsection}

\section{Introduction}

\subsection{Protein-based materials}

% Introduce proteins as materials - a sort of history survey of bio-based
% functional material and their corresponding various application fields.

Genetically engineered proteins have made a significant impact in biomaterials
research, enabling the fabrication of polymers with monodisperse molecular
weights, a diverse set of side chain functionality, and well-defined secondary
structure elements.\cite{
Yoshikawa1994,Wang2001,Rabotyagova2009,Martin2009,Dinerman2002,Megeed2002,Haider2004,Wright2002,Banta2010,Link2003,Straley2009,Rabotyagova2011}
% --------------------------
\begin{figure}[h!] \centering \includegraphics[width=0.7\textwidth]{f_1_03}
    \caption[The application of genetic engineering to the synthesis of new
            biomaterials. The primary acid sequence is first encoded into a
            complementary sequence of DNA, constructed via enzymatic ligation or
            chemical synthetic methods. The resultant synthetic gene is then
            inserted into an appropriate expression vector, yielding a
            recombinant plasmid, to be transformed into a host organism in which
            protein expression is
            expected.]
        {The application of genetic engineering to the synthesis of new
            biomaterials. The primary acid sequence is first encoded into a
            complementary sequence of DNA, constructed via enzymatic ligation or
            chemical synthetic methods. The resultant synthetic gene is then
            inserted into an appropriate expression vector, yielding a
            recombinant plasmid, to be transformed into a host organism in which
            protein expression is
            expected.\cite{Tirrell1991}}\label{fig:protein_engineering_methods}
        \end{figure}
% --------------------------
Compared against a backdrop of predominant synthetic polymers, protein polymers,
particularly those generated via high-fidelity recombinant DNA technologies,
exhibit defined molecular weight, composition, sequence, and
stereochemistry.\cite{VanHest2001} While proteins are not alone amongst
engineered materials of biological origin, they exist as the intersection of
biological function and structure. The latter feature spans the gamut of length
scales, ranging from the nanoscopic to the macroscopic, with notable examples
including actin filaments in the intracellular milieu to the silk fibers visible
to the naked eye.\cite{Rabotyagova2009,Megeed2002,Nagarsekar2002} Their frequent
roles as both sensors and chemical catalysts are testaments to their functional
impact on nature.

% Begin to focus on example of proteins that have secondary structural motifs
% that abstract themselves to higher order structures, such as fibers.
% Generalize how these can span length scales from nanoscopic order to
% macroscopic order.  Why do these higher order structures typically form. 

Numerous examples exist of proteins extrapolating the assembly process beyond
the quaternary scale of folding. This arena of self-assembly is of particular
focus herein, requiring inquiry into the evidence of such phenomenon occurring,
and more importantly, how these self-assembly events occur mechanistically via a
through-line that spans multiple scales of structure. In the case of collagen,
secondary structure is established as extended polyproline II-like helix, but
further progresses to the self-association of monomers to form triple helices.
More interestingly, this tiered assembly, wherein the larger macromolecular
assemblies become dependent on the secondary structure of monomers imbues the
opportunity for sequence-dependent specificity of assemblies. Indeed this
inherent feature of these self-assembling systems has been leveraged by several
engineering design groups.\cite{Kar2009,Stahl2010a,Fletcher2012}

Similarly, resilin, the proteinaceous material responsible for the elasticity of
various insect wings and appendages, relies on a critical proline residue to
accommodate regular turns in the protein backbone that promote a molecular
spring able to survive innumerable duty cycles.\cite{Rauscher2006} In the case
of elastin-based peptides, secondary structure is established as a preliminary
phase of assembly. The prominent theory in the field dictates that the higher
order self-assembly of this class of peptides is governed by the hydrophobic
collapse of the macromolecule. In certain cases, this results in a macroscopic
network of peptides to yield either a gel or colloid of
coacervates.\cite{Urry1991,Li2002a,Li2001,Urry1986,Urry1988}
% --------------------------
\begin{figure}[h!] \centering \includegraphics[width=0.8\textwidth]{f_1_17}
    \caption[Model of silk processing of \latin{B. mori} silk, demonstrating how
        macroscopic order originates from nanoscopic self-assembly processes.
        Shown in the figure, these silk constructs are composed of a repeating
        block architecture defining regions of hydrophobicity and
    hydrophilicity. This results in a bottom-up ordered assembly of the protein,
yielding higher order stable structures. This example demonstrates the promise
of engineering ordered materials from protein polymers.]{Model of silk
    processing of \latin{B. mori} silk, demonstrating how macroscopic order
    originates from nanoscopic self-assembly processes. Shown in the figure,
    these silk constructs are composed of a repeating block architecture
    defining regions of hydrophobicity and hydrophilicity. This results in a
    bottom-up ordered assembly of the protein, yielding higher order stable
    structures. This example demonstrates the promise of engineering ordered
    materials from protein polymers.\cite{Jin2003}}\label{fig:silk_structure}
\end{figure}
% --------------------------
Silk is another prominent natural protein on which novel biomaterial design is
based. It exists as an example of a heterogeneous mixture of structured and
unstructured secondary structures, giving rise to an example of a natural liquid
crystal material. $\beta$-sheets dominate the ordered features of silk variants,
giving rise to high packing density and ${\beta}$-crystallinity, which is
required for mechanical function. Interspersed, however, within the silk
structure are hydrophilic regions to control water
content.\cite{Bini2004a,Jin2003,Knight2002} Figure \ref{fig:silk_structure}
conveys the supramolecular ordering that occurs with natural silk assembly,
giving rise to macroscopic materials.

\subsection{Block protein copolymers}

Despite the well-known terminology associated with block copolymers, the
intentional generation of peptide-/protein-based block copolymers is still
limited in precedence. Only a few groups have demonstrated an application of the
polymeric motif with the use of genetic
engineering.\cite{Higashiya2007,Lim2008a,Chen2008a,Xu2005,Osborne2008} These and
other emerging group efforts leverage recombinant DNA technology to obtain precise
control over protein primary sequence, length and block distributions. 
The extent of predictability of the physicochemical properties of such block
copolymers is still elusive, and to-date, appears to require a case-by-case
empirical investigation. Studies of independent block components are, however,
carried out extensively, such as in the case of silk-, elastin-like peptide-,
coiled-coil protein-based fusions, establishing behavioral trends based on salt
concentration and primary sequence.\cite{Cho2008,Nuhn2008,Reiersen1998} In
addition, the interdependence of biophysical properties of disparate
self-assembling domains as they exist as part of fusion constructs is still
being probed. 

\subsubsection{Fusion Proteins}

While groups have devoted a great body of work to the study of block copolymers
in order to control structural and mechanical properties,\cite{Rabotyagova2009}
fusion polymers between functional macromolecules (i.e. enzymes and sole
binders) and structurally resilient motifs, such as silk, elastin, and
coiled-coils are just beginning to be developed via these
techniques.\cite{Kim2008a,Meyer1999,Wheeldon2009} A prime example is the fusion
of self-assembling proteins to host proteins to aid in the purification of the
host protein. The work by Meyer and Chilkoti pertaining to the fusion of
elastin-like peptides to host proteins for the high-throughput, rapid
purification via inverse-temperature cycling, by virtue of the elastin-like
peptide's ability to self-assemble and become insoluble upon mild
heating.\cite{Meyer1999} The work by Kim and Chilkoti further investigate the
effects of the host on the physicochemical properties of the elastin-like
peptide purification tag, demonstrating that the thermoresponsiveness of the
elastin-like peptide domain can be modulated via allosteric actuation of the
host protein.\cite{Kim2008a} Banta \latin{et al.} have adapted the unexpected
outcome of calcium-responsive insolubility amongst concatemers of the C-terminal
cap of the CyaA repeat-in-toxin domain to the non-chromatographic purification
of green fluorescent protein, maltose binding protein, and alcohol
dehydrogenase.\cite{Shur2013}

\subsubsection{Elastin-like peptides}

% Introduce the concept of supramolecular assembly.
Spawning from the seminal work of Dan Urry, elastin-like peptides (ELPs) have
been thoroughly investigated and modified for engineering purposes; the Chilkoti
group pioneered the fusion of ELPs to other proteins for the purpose of
purification via inverse temperature cycling.\cite{Meyer1999} A plethora of
additional applications of ELP have been developed, leveraging in large part its
lower critical solution temperature (LCST) behavior.\cite{Urry2002} For example,
in taking advantage of the enhanced permeability and retention (EPR) effect,
groups have chosen ELPs to trigger the assembly of nanoscopic coacervates
\latin{in vivo} that are capable of being entrained in leaky tumor
tissue.\cite{Nakayama2010,DeLasHerasAlarcon2005} 

ELPs consist of oligomeric repeats of a pentapeptide motif:
Val-Pro-Gly-Xaa-Gly, wherein the Xaa ``guest residue'' is any amino acid except
proline. These classes of peptides have been shown to adopt a ${\beta}$-spiral
secondary structure as they undergo the thermoresponsive transition, adopting 
more ordered structure upon heating as opposed to thermal denaturation to give
rise to disorder.\cite{Urry1988,Urry1986,Urry1985}
% --------------------------
\begin{figure}[h!] \centering \includegraphics[width=0.8\textwidth]{f_1_18}
    \caption[The hydrophobic
        collapse of an elastin-like peptide upon heating, as shown by molecular
        dynamics simulations. An elastin main chain is shown in red.
        Elastin-associated water molecules are depicted in blue and water
        molecules expelled at \SI{42}{\celsius} is shown in magenta. The
        non-polar solvent-accessible surface area at \SI{10}{\celsius}, below
        the transition temperature, comprises roughly \SI{80}{\percent} of the
        total solvent-accessible surface area, hypothesized to constitute the
        driving force for hydrophobic collapse and the lower-critical solution
        temperature transition phenomenon. Simulations such as these further
        reinforce the association of this collapse and aggregation event with
        significant ${\beta}$-turn/${\beta}$-spiral secondary/tertiary structure
        formation.]{The hydrophobic
        collapse of an elastin-like peptide upon heating, as shown by molecular
        dynamics simulations. An elastin main chain is shown in red.
        Elastin-associated water molecules are depicted in blue and water
        molecules expelled at \SI{42}{\celsius} is shown in magenta. The
        non-polar solvent-accessible surface area at \SI{10}{\celsius}, below
        the transition temperature, comprises roughly \SI{80}{\percent} of the
        total solvent-accessible surface area, hypothesized to constitute the
        driving force for hydrophobic collapse and the lower-critical solution
        temperature transition phenomenon. Simulations such as these further
        reinforce the association of this collapse and aggregation event with
        significant ${\beta}$-turn/${\beta}$-spiral secondary/tertiary structure
        formation.\cite{Li2001}}\label{fig:elastin_diagram}
    \end{figure}
% --------------------------
Elastin-like peptides have been known to undergo a thermoresponsive coacervation
event upon heating since its early pioneering studies by
Urry,\cite{URRY1974,Urry1985}, and the dominant theory pertaining to the
mechanism of LCST has come to be its hydrophobic collapse wherein upon raising
the temperature of these proteins in water, the more ordered water surrounding
the hydrophobic moieties becomes less ordered bulk water; this positive entropy
change is larger in magnitude than the negative entropy change that arises
because the polypeptide region of the system becomes more
ordered (Figure \ref{fig:elastin_diagram}).\cite{Urry1993} This theory has further been supported by numerous
simulations produced by the Daggett Group.\cite{Li2001,Li2003,Li2001a,Li2002a}

\subsubsection{Cartilage oligomeric matrix protein}
\label{sec:comp}

Cartilage oligomeric matrix protein (COMP) is a non-collagenous matrix protein
found in the cartilaginous extracellular matrix, discovered in the early 1990's,
first isolated and characterized in a native form from Swarm rat
chondrosarcoma.\cite{Morgelin1992} It is a member of the thrombospondin (TSP)
gene family of extracellular glycoproteins, found mainly in articular cartilage,
tendon, and
ligaments.\cite{Adams2001,Smith1997,Muller1998,Hedbom1992,Oldberg1992} Early
SDS-PAGE and electron microscopy studies indicated the ability of COMP to
pentamerize (Figure \ref{fig:COMP_EM_1}).\cite{DiCesare1995,Morgelin1992} 
% --------------------------
\begin{figure}[h!] \centering \includegraphics[width=0.6\textwidth]{f_2_09}
    \caption[COMP revealed by glycerol spraying and rotary
        shadowing with electron microscopy. Micrographs sampled from the 1995
        work by DeCesare \latin{et al.} clearly displays the five-armed
        molecules of COMP, demonstrating the ability of the macromolecule to
        self-assembling about a central core, composed of and stabilized by the
        N-terminal coiled-coil domain. Panel (A) shows low-magnification
    micrographs. Panel (B) shows high-magnification micrographs.]{COMP revealed by glycerol spraying and rotary
        shadowing with electron microscopy. Micrographs sampled from the 1995
        work by DeCesare \latin{et al.} clearly displays the five-armed
        molecules of COMP, demonstrating the ability of the macromolecule to
        self-assembling about a central core, composed of and stabilized by the
        N-terminal coiled-coil domain. Panel (A) shows low-magnification
    micrographs. Panel (B) shows high-magnification micrographs.\cite{DiCesare1995}}\label{fig:COMP_EM_1} \end{figure}
% --------------------------
The structure of the coiled-coil domain received heightened attention in the mid
to late 1990's for its channel-like structure. Work by Efimov and Malashkevich
revealed the dimensions of the coiled-coil domain, as well as hydrophobic
channel situated at the center of the pentameric ${\alpha}$-helical
bundle.\cite{Efimov1996,Malashkevich1996a} Structurally, the ${\alpha}$-helical
domain forms a homopentameric, parallel, left-handed coiled-coil with an average
length of \SI{70}{\angstrom} and an average outer diameter of approximately
\SI{30}{\angstrom}. The sequence obeys a conventional ${\alpha}$-helical heptad
motif pattern, with \emph{a} and \emph{d} positions predominantly occupied by
aliphatic, non-polar residues.\cite{Burkhard2001} 
% --------------------------
\begin{figure}[h!] \centering \includegraphics[width=0.8\textwidth]{f_1_21}
    \caption[Dimensions of the COMP coiled-coil domain, shown to bind small
        hydrophobic molecules such as two vitamin \ch{D3} molecules depicted
        here. When assembled as a pentamer, the complex forms a hydrophobic pore
        space with an estimated length-wise dimension of \SI{73}{\angstrom} and
    an inner diameter of \SIrange{2}{6}{\angstrom}. Structure determination was
conducted via X-ray crystallographic studies performed by Ozbek.]{Dimensions of
    the COMP coiled-coil domain, shown to bind small hydrophobic molecules such
    as two vitamin \ch{D3} molecules depicted here. When assembled as a
    pentamer, the complex forms a hydrophobic pore space with an estimated
    length-wise dimension of \SI{73}{\angstrom} and an inner diameter of
    \SIrange{2}{6}{\angstrom}. Structure determination was conducted via X-ray
    crystallographic studies performed by
    Ozbek.\cite{Ozbek2002}}\label{fig:comp_dimensions}
\end{figure}
% --------------------------
This later inspired work by Ozbek and Guo to study the putative role for COMP as
a storage and delivery protein for regulatory molecules in bone metabolism,
complexing with known transcriptional cofactors such as vitamin \ch{D3},
\iupac{all-\trans-retinol}, \iupac{all-\trans-retinoic acid}, and
benzene.\cite{Guo1998,Ozbek2002} Even still, interest in the binding
capabilities of coiled-coil domain of COMP has recently reemerged in the groups
of Stetefeld and Montclare, who have published data on the affinity of COMP for
fatty acids and curcumin, respectively, arguably with more intent on engineering
the COMP coiled-coil domain as an intentional
complexer.\cite{McFarlane2012,Gunasekar2009} These as well as Montclare
have shown the COMP coiled-coil to demonstrate dissociation constants in
the
\SIrange[scientific-notation=true,retain-unity-mantissa=false]{1e2}{1e-1}{\micro\moLar}
range.\cite{Haghpanah2010,Guo1998}

\subsubsection{Block ELP-COMP Protein Polymers}

Three protein-based block copolymers were previously developed and studied by
the Montclare group,\cite{Haghpanah2010,Haghpanah2009} for their physicochemical
properties. The copolymers were generated as a fusion of the aforementioned ELP
and COMPcc proteins, with their specific primary sequence shown in Figure
\ref{fig:pff_proteins}. These protein fusions were generated with the intention
of benefiting from the coacervation formation of ELP as well as the binding
capabilities of COMPcc. In the assessment of the block copolymers, it was
observed that there existed an unexpected dependence of the physicochemical
properties of the fusions on their block ordering (EC vs. CE) as well as their
block number (triblock vs. diblocks), with each of the three constructs
demonstrating distinct degrees of the expected properties (i.e.
thermoresponsiveness, binding affinity, extent of ${\alpha}$ and ${\beta}$
secondary structure).
% --------------------------
\begin{figure}[h!]
    \centering
    \includegraphics[width=0.8\textwidth]{f_1_01}
    \caption{Scheme indicating the three block copolymer proteins that were
        previously generated, via recombinant DNA techniques, and incorporated
        with \iupac{\para-fluorophenylalanine} (\emph{p}FF) via residue-specific
        incorporation in this work. The primary sequence and cartoons of
        the hypothetical construct architectures for (a) \emph{p}FF-EC, (b)
        \emph{p}FF-CE, and (c) \emph{p}FF-ECE are shown. The five phenylalanine
        residues of each elastin region, depicted in black, are replaced with
        \emph{p}FF under a hypothetical incorporation efficiency of
        \SI{100}{\percent}. The alpha-helical COMP coiled-coil domain is
        depicted in red. The His-tag domain, defined by six histidine residues,
        is depicted in pink at the N-terminus of each construct.
    }
    \label{fig:pff_proteins}
\end{figure}
% --------------------------
\subsection{Incorporation of non-natural amino acids}

While the physicochemical properties of protein-based materials have been
modulated with extraordinary capacity and range - in large part due to
the non-covalent interplay of amino acid residues - they are still limited by
the constrained 22 amino acids that define functional group diversity.
This limitation has been addressed, and in many ways deprecated, by various
efforts in genetic engineering.

There exists two contemporary methods to biosynthetically incorporate
non-natural amino acids into proteins: site-specific incorporation and
residue-specific incorporation.

\subsubsection{Site-specific incorporation methods} 

% Document the example of site-specific incorporation of azido-containing amino
% acids into polypeptides.

Early work in the mechanistic studies, and resultant ``hacking'' of the
components of the natural protein synthesis machinery, yielded the eventual
maturity of incorporation techniques.  Site-specific incorporation has been a
viable strategy for the incorporation of non-natural amino acids into
polypeptides since the late 1990's, yet its development stemmed from
advancements occurring as early as the late 1970's, with the work of 
Hecht on the chemical aminoacylation of tRNA and subsequent work on the
misacylation of tRNA{$^\text{Phe}$}.\cite{Heckler1984,Hecht1978} This later
fostered incremental efforts of Brunner in the charging of tRNA{$\text{Phe}$}
with \iupac{\L-4'-[3-(trifluoromethyl)-3\H-diazirin-3-yl]}phenylalanine, which
still required the chemically misaminoacylation of tRNA in the presence of T4
RNA ligase and ATP.\cite{Baldini1988}
% --------------------------
\begin{figure}[h!] \centering \includegraphics[width=0.6\textwidth]{f_1_05}
    \caption[Strategy for the site-specific incorporation of unnatural amino
    acids. This approach replaces the codon that encodes the amino acid of
    interest with the ``blank'' nonsense codon, TAG, by site-directed
    mutagenesis. The corresponding suppressor tRNA is chemically aminoacylated
    \latin{in vitro} with the unnatural amino acid. This affords the insertion
    of the prescribed amino acid into the protein at the targeted site upon
    addition of the aminoacylated tRNA to the \latin{in vitro} system.]{Strategy for the site-specific incorporation of unnatural amino
    acids. This approach replaces the codon that encodes the amino acid of
    interest with the ``blank'' nonsense codon, TAG, by site-directed
    mutagenesis. The corresponding suppressor tRNA is chemically aminoacylated
    \latin{in vitro} with the unnatural amino acid. This affords the insertion
    of the prescribed amino acid into the protein at the targeted site upon
    addition of the aminoacylated tRNA to the \latin{in vitro} system.
    \cite{Noren1989}}\label{fig:schultz_schema} \end{figure}
% --------------------------
With the seminal efforts of the Schultz \latin{et al.}\cite{Noren1989} shortly
following these earlier bodies of work, it is now possible to redefine the
genetic code. This work is based on the premise of directing a suppressor tRNA
against a ``blank'' nonsense codon. Initial works from this group involved the
chemical aminoacylation of the suppressor tRNA \latin{in vitro} (Figure
\ref{fig:schultz_schema}), but later efforts resulted in more complete
biosynthetic methods.
% --------------------------
Years later, in 2001, the Schultz group reported on the development of a unique
transfer RNA (tRNA)/aminoacyl-tRNA synthetase pair that were genetically
engineered to exist as orthongal complments to their endogenous analogues, as
well as associate the encoding for the amber nonsense codon with a non-natural
amino acid - \iupac{\ortho-methyl-\L-tyrosine}. The group was able to thus
incorporate a non-natural, synthetic amino acid with notably high fidelity
(99\%).\cite{Wang2001} This method relies on the translation suppression of
endogenous amber nonsense codons, which is addressed by the orthogonal nature of
the system's design. Similarly, Kwon and coworkers have been able to reassign
one of the degenerate codons (UUU) that correspond to phenylalanine to encode
for \iupac{2-naphthylalanine}.\cite{Kwon2003}

This technique has been employed in various contexts, ranging from the study of
the structure-function relationship of enzymes and receptors, to the rationale
design of proteins with higher activities or structural stabilities.
%How site-specific incorporation has been used to probe structure-function
%relationships

%How site-specific incorporation has been used to rationally design proteins

%Other groups have pushed the envelope still further with this technique,
%reimagining the genetic code itself.

\subsubsection{Global incorporation methods} 

% Include an argument in support of residue-specific incorporation via
% endogenous tRNA/aminoacyl-tRNA synthetase

Global incorporation of non-natural amino acids is further made possible by the
particular technique of residue-specific substitution. This technique is
typically characterized by a stochastic substitution of all amino acids of a
particular identity within a given sequence. It is made possible by two
prominent factors: 1) the promiscuity and relaxed specificity of aminoacyl-tRNA
synthetase endogenous to certain expression hosts, and 2) the existence of
auxotrophic expression hosts. Tirrell \latin{et al.} have modified \latin{E.
coli} as auxotrophic strains for Phe as well as Leu residues, allowing for the
possibility for downstream incorporation efforts, such as the incorporation of
both \iupac{(2\S,3\R)-4,4,4-trifluorovaline} and
\iupac{5,5,5-trifluoroisoleucine} into basic leucine zipper, coiled-coil
peptides derived from GCN4,\cite{Son2006} and the incorporation of monofluorinated Phe
residues into the histone acetyltransferase protein,
tGCN5.\cite{Voloshchuk2009} Generation of these auxotrophic strains, however,
requires a body of work in and of itself.\cite{Yoshikawa1994,Son2006} Figure
\ref{fig:chorismate_pathway} shows the pathway that was modified by Tirrell
\latin{et al.} in 1994 to give rise to the Phe auxotroph strain applied in this
work. The methods for incorporation have since matured and have become more
widely known.\cite{Hammill2007}.
% --------------------------
\begin{figure}[h!] \centering \includegraphics[width=0.7\textwidth]{f_1_02}
    \caption[Terminal pathway
        of phenylalanine biosynthesis, which was silenced by
        Tirrell \latin{et al.} in 1994 to yield the phenylalanine auxotroph
        designated AF. The auxotroph was generated from
        \latin{E.coli} strain BL21-(DE3) by P1-mediated transduction using a
        donor strain (NK6024) which had been transposed by TN10 at the
        \latin{pheA} region in the bacterial chromosome.  The transposon
        mutagenesis results in loss of the chorismate mutage-prephenate mutase
        activity via recombination events on the
chromosome.]{Terminal pathway
        of phenylalanine biosynthesis,\cite{Im1971} which was silenced by
        Tirrell \latin{et al.} in 1994 to yield the phenylalanine auxotroph
        designated AF.\cite{Yoshikawa1994} The auxotroph was generated from
        \latin{E.coli} strain BL21-(DE3) by P1-mediated transduction using a
        donor strain (NK6024) which had been transposed by TN10 at the
        \latin{pheA} region in the bacterial chromosome.  The transposon
        mutagenesis results in loss of the chorismate mutage-prephenate mutase
        activity via recombination events on the
chromosome.}\label{fig:chorismate_pathway} \end{figure}
% --------------------------
\subsection{Fluorinated amino acids}
\label{sec:fluorinated_amino_acids}

% What's the earliest precedence found for fluorinated amino acids and the
% motivation behind their synthesis?

Multiple efforts have been attempted to incorporate non-natural amino acids into
proteins, specifically fluorinated amino
acids.\cite{Meng2007,Lee2004,Son2006,Li2010a,Woll2006} Interest in the
incorporation of fluorinated amino acids has been attributed to the
stabilization effects of fluorination in general, enhancing hydrophobicity of
fluorocarbons relative to hydrocarbons and the preference for
fluorocarbon-fluorocarbon
interaction.\cite{Woll2006,Lee2004,Marsh2009,Jackel2006,Yoder2002} Examples
include the incorporation of trifluoromethyl phenylalanine by way of a modified
tRNA-synthetase to ultimately attempt to study \latin{E.coli} proteins via
\textsuperscript{19}F NMR,\cite{Li2010a,Jackson2007} the incorporation of hexafluoroleucine into
Buforin and Magainin,\cite{Meng2007} the incorporation of hexafluoroleucine into
GCN4 coiled-coils,\cite{Lee2004} and the incorporation of
pentafluoro-phenylalanine\cite{Woll2006} (Figure
\ref{fig:fluorination_examples}).
% --------------------------
\begin{figure}[h!] \centering \includegraphics[width=0.7\textwidth]{f_1_19}
    \caption[Examples of incorporation of fluorinated, non-natural amino acids
        into bacterially expressed proteins.
        % --------------------------
        (A)
        \iupac{Trifluoromethyl-\L-phenylalanine} incorporated into
        \latin{E.coli} proteins, nitroreductase (NTR) and histidinol
        dehydrogenase (HDH), by way of a modified aminoacyl-tRNA
        synthetase.
        % -------------------------- 
        (B)
        Incorporation of \iupac{hexafluoro-\L-leucine} into anti-microbial
        peptides, buforin and magainin, resulting in enhanced activity and
        protease stability.
        % --------------------------
        (C) Structural stability of GCN4 coiled-coils by way
        of incorporation of \iupac{hexafluoro-\L-leucine} into the hydrophobic
        core region, as evidenced by an increase in ${\Delta\Delta G}$.
        % --------------------------
        (D)
        Incorporation of \iupac{pentafluoro-\L-phenylalanine} into the non-polar
        core of chicken villin headpiece sub-domain (cVHP) defined by three
        phenylalanine residues.]
        {Examples of incorporation of fluorinated, non-natural amino acids
        into bacterially expressed proteins.
        % --------------------------
        (A)
        \iupac{Trifluoromethyl-\L-phenylalanine} incorporated into
        \latin{E.coli} proteins, nitroreductase (NTR) and histidinol
        dehydrogenase (HDH), by way of a modified aminoacyl-tRNA
        synthetase.\cite{Jackson2007}
        % -------------------------- 
        (B)
        Incorporation of \iupac{hexafluoro-\L-leucine} into anti-microbial
        peptides, buforin and magainin, resulting in enhanced activity and
        protease stability.\cite{Meng2007}
        % --------------------------
        (C) Structural stability of GCN4 coiled-coils by way
        of incorporation of \iupac{hexafluoro-\L-leucine} into the hydrophobic
        core region, as evidenced by an increase in ${\Delta\Delta
        G}$.\cite{Lee2004}
        % --------------------------
        (D)
        Incorporation of \iupac{pentafluoro-\L-phenylalanine} into the non-polar
        core of chicken villin headpiece sub-domain (cVHP) defined by three
        phenylalanine residues.\cite{Woll2006}}
        \label{fig:fluorination_examples}
    \end{figure}
% --------------------------
%\subsection{Fluorous effect}
%For submission in draft 2

In addition to the additional molecular control exemplified by the physical
properties of fluorinated amino acids and their incorporation in to
biomacromolecules, yet another potential benefit lies with the usage of
fluorinated compounds as \textsuperscript{19}F MRI contrast agents.
\textsuperscript{19}F MRI has been proven as a feasible method of imagining for
the last 20 years,\cite{Srinivas2010,Nelson1985} with its application relying on
several key features of the \textsuperscript{19}F nucleus itself:
% --------------------------
(1) comparable sensitivity to that of \textsuperscript{1}H;
(2) nearly \SI{100}{\percent} natural abundance;
(3) not present in natural biological tissues, and the amounts that do exist do
not contribute to background signal by virtue of short ${T_{2}}$ values
(4) marginal (~\SI{6}{\percent}) difference in resonance frequency from that of
\textsuperscript{1}H;
(5) a broad chemical shift and ${T_{1}}$ sensitivity to oxygen tension, lending
to \latin{in vivo} sensor applications; 
(6) blood substitutes based on perfluorocarbons (PFCs) have already been studied
for their biodistribution and toxic properties.
% --------------------------
The clear benefit to a path-controlled delivery agent, such as those based on
elastin-like peptides, exists from the incorporation of \textsuperscript{19}F,
potentially allowing for the tracking of the delivery agent \latin{in vivo}.
Indeed, \textsuperscript{19}F contrast agents, such as CS-1000 (Celsence, Inc.,
USA), have been used already to track dendritic cells \latin{in vivo} using MRI
and MR spectroscopy.\cite{Bonetto2010} These methods, in contrast to iron
oxide-based contrast agents, and Gd\pch[3]- and Mn\pch[2]-based contrast agents,
rely on spin density detection modes, which allow for quantitative measurements
to be made.\cite{Srinivas2010} The challenge lies with obtaining the complete
incorporation of fluorinated substitutes into a given protein construct and
ensuring that the physicochemical properties of the protein that makes it an
ideal delivery agent are not significantly and/or detrimentally altered.

\subsection{Scope of work}

The primary goals of this work were to evaluate the incorporation of
the fluorinated amino acid, \iupac{\para-fluorophenylalanine}, into protein block
copolymers that were previously developed, and to evaluate the effects of this
incorporation on the physicochemical properties of the polymers.
% --------------------------
It is hypothesized that the substitution of fluorine onto the guest residue of
the elastin-like peptide block sequence will significantly affect the behavior
of the bulk polymer, supported by the current understanding of elastin-like
peptides, elastin-like peptide fusions, and precedent observations of
fluorinated biomacromolecules. Validation of the physicochemical properties of
fluorinated variants of these particular block copolymers will further their
advancement toward MRI applications hinging upon the incorporation of
\textsuperscript{19}F nuclei. \textsuperscript{19}F MRI has been used to trace
monofluorinated dyes to detect the presence of amyloid ${\beta}$ plaques
\latin{in vivo}.\cite{Higuchi2005} Fluorinated compounds have also been used
more recently to track dendritic cells  \latin{in vivo}.\cite{Bonetto2010}
Proteins, biosynthetically incorporated with fluorinated amino acids, such as
that documented in the work by Jackson,\cite{Jackson2007} have been studied with
\textsuperscript{19}F NMR for characterizing the structure and reactivity of the
protein \latin{in vivo}.  These are two precedent examples of a growing many
that demonstrate the potential applicability of these fluorinated block
copolymers for biomedical imaging applications.

% Combining the notion of innocuous and non-invasive tracing of small molecule
% delivery agents based on elastin-like peptides, localized to sites of disease
% via the EPR effect.

% Reference the particular of the following works in an effort to draft a
% rational hypothesis of what will happen upon incorporation. Then justify why
% CD, turbidity, and microrheology is used to interrogate these effects.
% vide supra List
% Guest residue effects by Urry.
% Chilkoti's block influence work.
% Fluorous effect.

Incorporation of \emph{p}FF into the block copolymer constructs was accomplished
via residue-specific incorporation, relying on the endogenous
\textsuperscript{Phe}tRNA and aminoacyl-\textsuperscript{Phe}tRNA synthetase of
AF-IQ Phe auxotrophic cells. This method of incorporation has been previously
validated by Montclare and
Tirrell.\cite{Voloshchuk2009,Panchenko2006,Baker2011,Sharma2000,Yoshikawa1994}
To probe the effects on the physicochemical behavior of these block copolymers,
three primary experiments were conducted, spanning various scales of structure -
circular dichroism spectroscopy, UV/Vis spectroscopy, and microrheology.
Circular dichroism was implemented in order to interrogate the effects of
incorporation on the secondary structure of the fusion proteins as a function of
temperature, thereby probing the affects of \emph{p}FF on the transitional
nature of the protein. UV/Vis spectroscopy was implemented to roughly quantify
the extent of macroscopic assembly occurring as a result of temperature-induced
transitions, the mechanical nature of which was probed by microrheology. 

\subsubsection{Circular Dichroism} Circular dichroism has been used extensively
in the investigation of proteins, providing a relatively rapid method for
structure elucidation with minimal sample requirements (i.e. mass, volume,
concentration, and matrix). It lacks the capacity, however, for exact structural
determination, which has become readily possible amongst a myriad of classes of
proteins via high resolution NMR and X-ray diffraction analysis.\cite{Adler1973}
Nevertheless, circular dichroism has been utilized to fingerprint protein
secondary structure since the 1960's, including one of the foundational studies
on \iupac{poly-\L-lysine} by Greenfield.\cite{Greenfield1969} The method
of analysis has continued to mature across decades of application to various
biomolecules, yielding advancements in the computationally-aided deconvolution
of
spectra,\cite{Andrade1993,Sreerama1993,Manavalan1987,Provencher1981,Johnson1999}
and expansion into the far-UV region of CD
spectroscopy.\cite{Brahms1980,Wallace2009,Miles2006} It is a technique of
particular relevance as it has been applied extensively to both classes of
blocks of the block copolymers investigated in this work, providing a nominal
reference structures on which to base the interpretation of structural
alterations. Variants of the elastin-like peptide have undergone extensive CD
spectroscopic characterization, including investigations on the affects of motif
pattern modifications,\cite{Nuhn2008,Kim2005} solvent conditions,\cite{URRY1974}
as well as polymer length.\cite{Nuhn2008,Urry1986,Reiersen1998} While the larger
cartilage-oligomeric matrix protein has not been extensively studied for its
secondary structure via CD spectroscopy, its coiled-coil region has been
isolated and studied for its secondary structural conformation with CD
spectroscopy.\cite{Gunasekar2009,Terskikh1997} More recently, the natively
constructed block copolymers themselves were characterized with CD
spectroscopy.\cite{Haghpanah2009}

\subsubsection{UV/Vis spectroscopy}
It has long been the convention to monitor the coacervation of materials that
demonstrate a lower critical
solution temperature point via their temperature-resolved absorption in the
range of \SIrange{300}{400}{\nm}. Furthermore, the lower critical solution
temperature behavior, a key characteristic of elastin-like peptides,
is typically qualified by the point of half-maximal turbidity.\cite{Urry1993}

\subsubsection{Microrheology}
While several methods exist to evaluate the mechanical properties of fluids,
colloidal suspensions, and polymeric gels, microrheological techniques are
necessary to evaluate samples that are limited in volume and material
homogeneity. Classical oscillatory plate and capillary cell rheometers could not
accommodate the anticipated nature of block protein polymers, with expectedly
low viscoelastic moduli and low response frequencies. Micorheological methods,
however, could be employed to observe the mechanical properties of dilute
supramolecular assemblies. These methods include the application
fluctuation-dissipation theorem-based (FDT)
microrheology,\cite{Gittes1997,Mason1997,Mason1997a} two-point microrheology
(TPM),\cite{Crocker2000,Levine2000,Qiu2004,Levine2001} and atomic force
spectroscopy.\cite{Salman2002,Caspi2002} Herein, we employ a form of FDT-based
microrheology, which may be accomplished via a myriad of instrumental
techniques, including laser-deflection tracking,\cite{Gittes1997,Mason1997}
optical tweezers,\cite{Starrs2003,Addas2004} dynamic light
scattering,\cite{Dasgupta2002,Popescu2002} and image-based particle
tracking\cite{Valentine2004,Crocker2000,Chen2003}- all methods toward an effort
to quantify the Brownian motion of particles suspended in the rheological fluid
of interest.

The particular application of particle tracking documented herein is based on
the work and source code distributed by Weitz and Mason for particle tracking
microrheology of complex fluids.\cite{Mason1997} The technique assumes a
homogeneous isotropic solution in which an ensemble of spherical particles are
suspended, and the motion of which is primarily governed by the Brownian motion
within the medium.  The technique further assumes that the continuum medium is
incompressible. Given these assumptions, the technique relates the viscoelastic
spectrum of the fluid medium, ${\tilde{G}(s)}$, to the unilateral Laplace
transform of the mean square displacement, ${\langle \Delta r^2(t) \rangle}$, of
the particles, via the Stokes-Einstein equation:

\begin{equation}
    \label{eq:stokes-einstein}
    \tilde{G}(s)=
    \frac{k_BT}{\pi a s \langle \Delta \tilde{r}^2(s) \rangle}
    \text{ (units : \si{\pascal})}
\end{equation}

where ${s}$ is the Laplace frequency and ${k_B}$ is the Boltzmann's constant. In
this work, ${\tilde{G}(s)}$ was computed from ${\Delta r^2(t)}$ obtained via
image-based particle tracking, \latin{vide infra}.

\section{Methods}

\subsection{Recombinant gene construction}

Plasmid vectors containing the genes for the block protein polymer fusions were
created in a modular fashion, described in a previously accomplished body of
work.\cite{Haghpanah2009} Briefly, PCR assembly of the COMP gene and
flanking linker regions yielded a pQE9 vector (Qiagen) containing the final
single block construct.\cite{Kwon2003} This was then digested with SalI and SacI restriction
enzymes, and gel purified for insertion into a pQE30 plasmid vector, yielding
pQE30/C. The 5-repeat elastin gene was PCR amplified from pUC19RGDELF, donated
by David Tirrell, and cloned into pQE30/C between various restriction sites, to
yield pQE30/EC (BamHI and SacI), pQE30/CE (SalI and HindIII), and pQE30/ECE (all
4 sites). Complete plasmid sequences are presented in Appendix
\ref{chap:seq_data}. 

\subsection{Biosynthesis}

The aforementioned plasmid vectors were transformed into the phenylalanine
auxotrophic \emph{E. coli} strain AF-IQ, previously created and donated by David
Tirrell \emph{et al}.\cite{Yoshikawa1994,Sharma2000} Cell growth methods were based on
previously established conditions for the incorporation of unnatural amino
acids, particularly \emph{p}FF, into target
proteins.\cite{Voloshchuk2009,Yoshikawa1994}

Plasmids were transformed into AF-IQ cells via electroporation using the
following parameters on an electroporator (BioRad, Inc.) The cells were then
plated onto a tryptic soy agar plate, 40 \si{\g\per\L}, supplemented with
ampicillin (200 \si{\ug\per\mL}) and chloramphenicol (35 \si{\ug\per\mL}), and
allowed to grow overnight at 37 \si{\celsius}. Single colonies were then picked
and used to inoculate starter cultures, grown in M9 minimal media, supplemented
with the following components:

%START_TABLE
%PRE-INDUCTION REAGENTS TABLE
\begin{table}[h!]
    \centering
\begin{tabular}{ ll }
  \hline
  \multicolumn{2}{c}{Component (Final Concentrations)} \\
  \hline

  ampicillin & \SI{200}{\ug\per\mL} \\
  chloramphenicol & \SI{35}{\ug\per\mL} \\
  thiamine hydrochloride & \SI{34}{\ug\per\mL} \\
  magnesium sulfate & \SI{1}{\milli\moLar} \\
  calcium chloride & \SI{0.1}{\milli\moLar} \\
  glucose & \SI{0.4}{\wtper} \\
  20 amino acids & \SI{100}{\ug\per\mL} \\

  \hline
\end{tabular}
\caption{Pre-induction M9 Supplement Specifications}
\label{tab:preinduction_recipe}
\end{table}
%END_TABLE

Cells were grown in M9 supplemented media in two stages - a pre-induction stage
and a post-induction stage - which are separated by a media shift step.  The
starter culture was used to inoculate a larger \SI{100}{\mL} culture, as
\SI{4}{\volper}, grown in pre-induction M9 media as per Table
\ref{tab:preinduction_recipe}.  The culture was then allowed to grow in a
\SI{1}{\L} flask for \SI{6}{\hour} at \SI{37}{\celsius}, shaking at
\SI{350}{\rpm}. The culture was subsequently harvested and pelleted via
centrifugation. The pellet was subjected to resuspension in a volumetric
equivalent of \SI{0.9}{\wtper} NaCl solution, followed by centrifugation.  This
``wash'' step was performed 2 times in an effort to deplete the media of
residual amounts of extracellular phenylalanine. The washed pellet was
resuspended in fresh M9 supplemented media.  The distinct difference between the
pre- and post-induction supplemental reagents pertain to the amino acids
constitution; the post-induction media does not contain phenylalanine, but
rather contains \SI{91.59}{\mg\per\mL} \iupac{\para-fluorophenylalanine},
prepared from a stock.  Cells in the new media were allowed to grow at the same
incubation conditions for \SI{30}{\minute}, at which time the media was
supplemented with \iupac{isopropyl \b-\D-1-thiogalactopyranoside} (IPTG,
\SI{100}{\ug\per\mL}).  Cells were harvested and stored at -20 \si{\celsius} for
subsequent lysis and purification.

\subsection{Protein purification}

All solutions used in the extraction and purification of recombinant proteins
were filtered through \SI{0.22}{\um} sterile filters. In addition, all solutions
were adjusted to the appropriate \pH with either \SI{37}{\percent} \ch{HCl} or
\SI{1}{\moLar} \ch{NaOH}. Cell pellets were thawed at \SI{4}{\celsius} and
immediately subected to an osmotic shock protocol adapted from methods set forth
by Berglund \emph{et al}.  for the preparation of cell lysate for immobilized
metal affinity chromatography (IMAC)
purification.\cite{Neu1965,Bae2000,Magnusdottir2009} Briefly, cells were gently
resuspended in a cold hypertonic sucrose solution, consisting of
\SI{50}{\milli\moLar} HEPES, \SI{20}{\wtper} sucrose, \SI{1}{\milli\moLar}
ethylenediaminetetraacetic acid (EDTA), \pH 7.9. For every \SI{10}{\mL} of
original culture media, the resultant pellet was resuspended in \SI{1}{\mL} of
sucrose solution. The cell suspension was then pelleted via centrifugation at
\SI{7000}{\gforce} for \SI{30}{\minute} at \SI{4}{\celsius}. The resultant
pellet was separated from its supernatant and gently resuspended in a cold
hypotonic solution consisting of \SI{5}{\milli\moLar} \ch{MgSO4}, with a
volumetric equivalent to that of the previous step, and incubated on ice for
\SI{10}{\minute}. The suspension was then centrifuged at \SI{4500}{\gforce} for
\SI{20}{\minute} at \SI{4}{\celsius}. The resultant pellet was separated from
its supernatant and stored at \SI{-20}{\celsius} for downstream lysis. 
% --------------------------
Osmotically shocked expression pellets were resuspended in denaturing lysis
buffer, consisting of \SI{50}{\milli\moLar} \ch{Na2HPO4}, \SI{6}{\moLar}
\ch{CO(NH2)2}, \pH 8.0. The pellet was resuspended in a 4:70 volumetric ratio
(buffer:culture volume) on ice. The cell lysate was then passed through a French
press (40K-cell; Thermo Scientific), processed at \SI{165.5}{\MPa}, and
collected on ice. The cold lysate was then centrifuged at \SI{40000}{\gforce}
for \SI{45}{\minute} at \SI{4}{\celsius}. The clarified lysate was then applied
to a \SI{5}{\mL} HiTrap IMAC FastFlow column (GE Life Sciences), charged with
\ch{CoCl2}, using an \"{A}KTA purifier system (GE Life Sciences). The lysate was
loaded with \SI{0}{\moLar} imidazole and washed with 10 column volumes (CV) of
the same buffer constitution. The protein was then eluted across 1 CV of
\SI{48}{\milli\moLar} imidazole and 3 CV of \SI{1}{\moLar} imidazole.  Purified
elutions were assayed for purity via sodium dodecylsulfate polyacrylamide gel
electrophoresis (SDS-PAGE) analysis and, contingent upon confirmation of
\SI{90}{\percent} purity, was dialyzed extensively against water with a 3500
MWCO regenerated cellulose membrane (SnakeSkin Dialysis Tubing, Pierce) and
subsequently freeze-dried prior to use in experiments. Polyacrylamide gels were
cast using ammonium persulfate and tetramethylethylenediamine-catalyzed
polymerization to obtain \SI{12}{\wtper} acrylamide gels. Gel lanes of pure
protein were used to quantitate post-purification yields, using an ImageQuant
gel imaging system (GE Life Sciences). Loaded mass of protein, from
\SI{10}{\uL}, was calibrated against a volumetric equivalent of Precision Plus
Protein Unstained Standard mixture (Bio-Rad).

\subsection{Compositional characterization}
\subsubsection{MALDI-ToF mass spectrometry}

Incorporation levels of \emph{p}FF were assessed via matrix-assisted laser desorption
ionization, time-of-flight mass spectrometry analysis on an Omniflex
spectrometer (Bruker Daltonics). Chymotryptic digests of purified proteins were
prepared by resuspending \SIrange[range-phrase=--]{50}{100}{\ug} of protein in
\SI{100}{\milli\moLar} Tris-HCl and \SI{10}{\milli\moLar} \ch{CaCl2} (\pH 8.0)
to a final concentration of \SI{1.25}{\ug\per\uL}. Sequencing grade chymotrypsin
(Promega) was added to the mixture to yield a final concentration of
\SI{2}{\wtperwt}. The digestion reaction was allowed to occur at \SI{25}{\celsius}
for \SI{20}{\hour} and subsequently stopped by adjusting the mixture to \pH 4 by
titrating with \SI{1}{\percent} trifluoroacetic acid (TFA). The protein digest
mixtures were desalted and concentrated by using ZipTip(\ch{C18}) silica resin
(Millipore).  Desalted sample mixtures were mixed with saturated solutions of
\iupac{\a-cyano-4-hydroxycinnamic acid} prepared in \SI{1}{\percent}
TFA/acetonitrile (2:1) in a 1:1 ratio. Samples were then spotted onto a 7x7
OmniFlex MALDI target (Bruker Daltonics) and allowed to dry under vacuum.
Relative estimates of peptide digest fragment quantities were conducted by
comparing relative peak intensities at different \emph{m/z} values. MS spectra
were calibrated against a standard mixture of peptides, provided by New England
Biolabs.

\subsubsection{Amino acid analysis}

Amino acid analysis was performed on purified protein samples that have
undergone hydrolysis. Hydrolysate samples were then analyzed on a Hitachi L-8900
pH amino acid analyzer at the W. M. Keck Foundation Biotechnology Resource
Laboratory at Yale University. Incorporation levels of \emph{p}FF were calculated based
on the measurements of normalized Phe levels in the samples.

\subsection{Physicochemical characterization}

\subsubsection{UV/Vis spectroscopy}

UV/vis spectroscopy was performed on a Cary 100 UV/vis spectrometer (Agilent) to
determine the inverse temperature transition point ({$T_t$}) of 
\SI{4}{\micro\moLar} protein prepared in \SI{10}{\milli\moLar} Gomori buffer,
\pH 8.0. The absorbance at \SI{350}{\nm} was measured as each protein sample was
heated from \SIrange{10}{65}{\celsius}, at a heating rate of
\SI{0.5}{\celsius\per\minute}.\cite{Chilkoti2002a}

\subsubsection{Circular dichroism spectroscopy}

Circular dichroism (CD) spectroscopy was carried out on a J-815 CD polarimeter
(Jasco) equipped with a PTC-423S single position Peltier temperature control
system and cuvette holder, counter-cooled with an Isotemp 3016S water bath
(Fisher Scientific) set to \SI{22}{\celsius}. Protein samples were prepared to
\SI{4}{\micro\moLar} in \SI{10}{\milli\moLar} Gomori buffer, by resuspending
lyophilized protein and loading into a Helma 218 quartz cuvette (\SI{500}{\uL}
capacity, \SI{1}{mm} path length). Immediately prior to loading into the
spectrometer, samples were degassed for \SI{10}{\minute} by sparging samples
with nitrogen. Operation and analysis parameters were adapted from existing
procedures.\cite{Haghpanah2009} Wavelength scans were performed while holding
temperature constant as the samples were heated from \SIrange{10}{65}{\celsius},
at a heating rate of \SI{1}{\celsius\per\minute}.  Wavelength scans were
performed from \SIrange{250}{190}{\nm} with a \SI{1}{\nm} step size. Wavelength
scans were collected as the average of 3 consecutive scans, with a preliminary
equilibration time of \SI{5}{\minute} prior to the first scan. All wavelength
scans were subtracted from the background of Gomori buffer.

Mean residue ellipticity, $[\theta]_{\mathrm{mrw}}$ was calculated from
expression \ref{eq:mre} below:

\begin{equation}
    \label{eq:mre}
    [\theta]_{\text{mrw}}=
    \frac{[\theta]_{\text{obs}}\cdot\text{MRW}}{10\cdot{C_\text{mg/mL}}\cdot{L}}
    \text{ (units : \si{\degtext\cm\squared\per\deci\mol})}
\end{equation}

where $[\theta]_{\text{obs}}$ is the observed ellipticity signal, MRW is the
mean residue weight (molecular weight divided by the number of residues), $L$ is
the path length of the cell (in cm), and $C_\text{mg/mL}$ is the protein's
concentration.\cite{Martin2008} MRE spectral data was smoothed using a
Savitsky-Golay low-pass filter, applied with SpectraManager software package
(Jasco), using a smoothing width of 25.\cite{Savitzky1964} Estimation of
secondary structure was performed computationally using the CDSSTR method, as
distributed by DICHROWEB (Birkbeck College of the University of
London).\cite{Whitmore2008} The method was applied with the SDP49 protein
reference set.\cite{Sreerama2000a,Sreerama2000b}

An implementation of the variable selection method,\cite{Manavalan1987} entitled
CDSSTR, was employed to supplement the interpretation of CD spectra. Singular
value decomposition of the initially seeded data set of basis proteins,
designated SDP48, resulted in a reduced data set, which then are averaged as
best fit basis spectra. Four metered rules are imposed to filter collections of
best fit spectra: the sum of fractions of structural content is between 0.95 and
1.05; each fraction is greater than -0.03; the RMS deviation between the
reconstructed and experimental CD is less than 0.25 ${\Delta\epsilon}$; helical
content is concluded from helical fraction estimates using the complete
reference set and the maximum/minimum helix fraction from the acceptable
solutions. The SDP48 reference set is comprised of CD data garnered from 43
soluble proteins and 5 denatured/disordered proteins. Those classified as
denatured/disordered are known to contain \SI{90}{\percent} unordered structures
and \SI{2}{\percent} of each of the other types of
structures.\cite{Sreerama2000b}

\subsubsection{Transmission electron microscopy}
\label{sec:TEM_method}

Prior to transmission electron microscopy (TEM), lyophilized protein samples
were dissolved in \SI{10}{\milli\moLar} Gomori buffer, pH 8.0 (filtered through
a \SI{0.22}{\um} filter to a final concentration of \SI{0.4}{\mg\per\mL}.
Samples were either incubated at \SI{4}{\celsius} or \SI{40}{\celsius} for
\SI{1.5}{\hour} prior to application onto a carbon-coated copper grid. Sample
grids were glow-discharged for \SI{1}{\minute} prior to applying samples.
samples were negatively stained, consistent with the adhesion drop method; a
drop of sample was deposited on a carbon film-coated 400-mesh Cu/Rh grid and was
allowed to dry at room temperature for \SI{5}{\minute}. The grids were then
stained with 4 drops of \SI{2}{\volper} uranyl acetate solution, pH 3.5. The
samples were then observed on a Philips CM12 tungsten emission transmission
electron microscope (FEI, Eindhoven, The Netherlands) at \SI{120}{\kV} equipped
with a Gatan 4k ${\times}$ 2.7k digital camera (Gatan Inc., Pleasanton, CA).
Samples were either applied directly from storage at \SI{4}{\celsius} or
pre-incubated at \SI{50}{\celsius} for \SI{30}{\minute}.

\subsubsection{Microrheology}

To interrogate the mechanical properties of the proteins, passive microrheology
was performed using an inverted Leica DM-138 IRB microscope with a 20$\times$
objective lens at \SI{22}{\celsius} and \SI{42}{\celsius}, maintained by a
Linkam temperature-controlled microscope stage. Samples were prepared from
lyophilized proteins by dissolving dry powder in deionized water at
\SI{4}{\celsius} to two concentrations, \SI{1.25}{\mg\per\mL} and
\SI{2.5}{\mg\per\mL}. \SI{2}{\uL} of a \SI{2}{\wtper} stock of
fluorescently-labeled amidated polystyrene beads (\SI{1.0}{\um} was added to
\SI{10}{\uL} samples of protein. Brownian motion of the beads was observed by a
Peltier-cooled video camera (QiCam) and digitally recorded.  Mean square
displacement (MSD) measurements of the beads, dispersed in the protein sol gels,
were determined using IDL software (Research Systems, Inc.) and an application
of microrheological processing subroutines set forth by Mason and
Weitz.\cite{Mason1997} The subroutines as well as the parameters which they are
fed, are documented in Appendix B. 

\section{Results}

\subsection{Biosynthesis of Fluorinated Block Copolymers}
\label{sec:biosynthesis_results}
%Cite comparative SDS-PAGE results
% --------------------------
\begin{figure}[h!] \centering \includegraphics[width=0.5\textwidth]{f_1_04}
    \caption[Confirmation of
        \emph{p}FF-EC, \emph{p}FF-CE, and \emph{p}FF-ECE proteins, SDS-PAGE of
        whole cell lysate samples (normalized to OD{$_{600}$} = 1.0) in the
        absence and presence of phenylalanine and \emph{p}FF. Arrows indicate
        protein over-expression band. The absence of the over-expression bands
        in lanes 1, 2, 5, 6, 9, and 10 indicates that the expression methodology
        does not suffer from ``leaky'' expression of protein, which would result
        in the significant contamination of Phe in the purified protein.
        Comparable intensities of paired bands in lanes +Phe/-\emph{p}FF and
        -Phe/+\emph{p}FF suggests that residue-specific incorporation can
        readily support the presence and incorporation of \emph{p}FF in the
    system.] {Confirmation of
        \emph{p}FF-EC, \emph{p}FF-CE, and \emph{p}FF-ECE proteins, SDS-PAGE of
        whole cell lysate samples (normalized to OD{$_{600}$} = 1.0) in the
        absence and presence of phenylalanine and \emph{p}FF. Arrows indicate
        protein over-expression band. The absence of the over-expression bands
        in lanes 1, 2, 5, 6, 9, and 10 indicates that the expression methodology
        does not suffer from ``leaky'' expression of protein, which would result
        in the significant contamination of Phe in the purified protein.
        Comparable intensities of paired bands in lanes +Phe/-\emph{p}FF and
        -Phe/+\emph{p}FF suggests that residue-specific incorporation can
        readily support the presence and incorporation of \emph{p}FF in the
    system.}
        \label{fig:expression_gel_pFF} \end{figure}
% --------------------------
Residue-specific incorporation of \emph{p}FF was accomplished using 
established techniques by Montclare \latin{et al.}, and previously by Tirrell
\latin{et al.}.\cite{Yoshikawa1994,Sharma2000,Voloshchuk2009} Three primary methods of
incorporation analysis were conventionally practiced at the time of this work:
SDS-PAGE, amino acid analysis (AAA), and mass spectrometry (MS). 
%Cite comparative works that use this overall approach of 3 methods.
This regimen of analysis for incorporation of non-natural amino acids has been
recently implemented by Montclare, Baker, and
Voloshchuk\cite{Voloshchuk2009,Panchenko2006,Baker2011}. In
addition, several groups have implemented AAA as part of their evaluation of novel
peptides. MALDI-ToF MS, due to its limited quantitative capacity, has been
implemented to a lesser extent, but still has been employed in some
studies.\cite{Taki2001}

SDS-PAGE analysis of \emph{E.coli} expression lysate samples were carried out to
confirm proper control of induction via the \emph{lac} operon of the pQE-30
expression vector, as well as the suppression of significant amounts of protein
expression in the absence of \iupac{phenylalanine}.  Figure
\ref{fig:expression_gel_pFF} indicates the sufficient control of expression so
as to achieve effectively sole incorporation of \emph{p}FF. The absence of
expression in the absence of IPTG and the absence of either Phe or \emph{p}FF
suggests a tight control of the inducible expression system. This is consistent
with the complementary design of the expression host/vector pairing - AF-IQ and
pQE-30 - of this implementation.

More specifically, in the presence of Phe or \emph{p}FF, a distinct band was
observed at \SI{25}{\kilo\dalton} for the diblocks and \SI{37}{\kilo\dalton} for
the triblock, confirming the production of protein polymers. Similar results
were obtained from previous studies, of equivalent differentials in band
intensities so as to conclude suppression of expression.\cite{Voloshchuk2009}

Biosynthetic yields for all three proteins were computed from the gel-based
quantitation of SDS-PAGE gels of purified protein. Overall yields from
expression and purification range from \SIrange{3}{6}{\mg\per\L} of
expression.
% --------------------------
\begin{figure}[h!]
    \centering
    \begin{subfigure}[b]{0.3\textwidth}
        \includegraphics[width=\textwidth]{f_1_09}
        \caption{\emph{p}FF-EC}
        \label{fig:A}
    \end{subfigure}
    \begin{subfigure}[b]{0.3\textwidth}
        \includegraphics[width=\textwidth]{f_1_10}
        \caption{\emph{p}FF-CE}
        \label{fig:B}
    \end{subfigure}
    \begin{subfigure}[b]{0.3\textwidth}
        \includegraphics[width=\textwidth]{f_1_11}
        \caption{\emph{p}FF-ECE}
        \label{fig:C}
    \end{subfigure}
    \caption{Inverted gray scale images of SDS-PAGE gels of total purified
    protein elutions, with accompanying quantitation in Table
    \ref{tab:yield_analysis}}\label{fig:purification_gels}
\end{figure}
% --------------------------
AAA and MS serve as orthogonal means to assess incorporation, with each method
providing its own merits and succumbing to its respective limitations. While AAA
provides high-fidelity quantitation of the amino acid composition of a protein,
it comes at the expense of complete hydrolysis of the sample and resultant loss
of primary sequence information. MS of enzymatic-digest fragments of the
protein, on the other hand, may only be considered quasi-quantitative under the
assumption of equivalent propensities of fragments to ionize, despite the
difference in primary structure and molecular weight; however, under this
assumption, it may be used to elucidate the relative incorporation levels of a
non-natural amino acid across different segments of the protein, allowing the
researcher to conclude a homogeneous dispensation of non-natural amino acid
throughout the primary sequence of the protein.
% --------------------------
%START_TABLE
%Expression Yield TABLE
\begin{table}[h!]
    \centering
\begin{tabular}{ lllllll }
  \hline
  Lane & 
  \multicolumn{3}{c}{Mass in Lane (\si{\ng})} & 
  \multicolumn{3}{c}{Mass in \SI{1}{\mL} aliquot (\si{\ug})} \\
  \hline
  &
  \emph{p}FF-EC &
  \emph{p}FF-CE &
  \emph{p}FF-ECE &
  \emph{p}FF-EC &
  \emph{p}FF-CE &
  \emph{p}FF-ECE \\ 
  \hline
  1 & 545 & 710 & 752 & 55 & 71 & 75 \\
  2 & 845 & 780 & 1118 & 85 & 78 & 112 \\
  3 & 948 & 711 & 948 & 95 & 71 & 95 \\
  4 & 650 & 506 & 783 & 65 & 51 & 78 \\
  5 & 459 & 350 & 592 & 46 & 35 & 59 \\
  6 & 330 & 282 & 481 & 33 & 28 & 48 \\
  7 & 264 & 212 & 395 & 26 & 21 & 39 \\
  8 & 231 & 167 & 344 & 23 & 17 & 34 \\
  9 & 172 & 156 & 268 & 17 & 16 & 27 \\
  \hline
  \multicolumn{4}{r}{Yield \si{\mg \per \L of expression}} & 4.44 & 3.87 & 5.68
\end{tabular}
\caption{Quantitation of protein yields from gels in Figure
    \ref{fig:purification_gels}.}
\label{tab:yield_analysis}
\end{table}
%END_TABLE
% --------------------------
%START_TABLE
%Incorporation Levels TABLE
\begin{table}[h!]
    \centering
\begin{tabular}{lcc}
  \hline
  \multicolumn{3}{c}{Incorporation Levels} \\
  \hline
  Protein Construct & AAA & MALDI-ToF MS \\
  \hline
  \emph{p}FF-EC & \SI{81.00(12)}{\percent} & \SI{89(5)}{\percent} \\
  \emph{p}FF-CE & \SI{77.75(29)}{\percent} & \SI{87(7)}{\percent} \\
  \emph{p}FF-ECE & \SI{88.85(100)}{\percent} & \SI{94(3)}{\percent} \\
  \hline
\end{tabular}
\caption{Incorporation levels as measured by Amino Acid Analysis (AAA) and
MALDI-ToF MS Analysis, accompanied by associated standard deviations for
respective measurements.}
\label{tab:incorporation_numbers}
\end{table}
%END_TABLE
% --------------------------
MALDI-ToF analysis of chymotryptic digests of each of the proteins was carried
out. Figures \ref{fig:MALDI_spectrum_EC}, \ref{fig:MALDI_spectrum_CE}, and
\ref{fig:MALDI_spectrum_ECE} delineate the main two fragments that result from
chymotrypsin digestion, which cleaves the protein at Phe residues.
Fortuitously, the enzyme appears to easily cleave sites that display \emph{p}FF
as well as Phe. Each spectrum indicates a shift in mass-to-charge ratio that
results from the shift imparted by an additional \SI{18}{\dalton}. Four regions
of interest on the MS spectra are highlighted, indicating expected fragments
from the digestion reaction, three of which derive from the COMP domain, and one
of which is the degenerate fragment of the elastin-like peptide domain.
% --------------------------
\begin{figure}[h!] \centering \includegraphics[width=0.95\textwidth]{f_1_06}
    \caption{MALDI-ToF spectrum of \emph{p}FF-EC with computed extents of
incorporation based on peak intensities.} \label{fig:MALDI_spectrum_EC} \end{figure}
% --------------------------
\begin{figure}[h!] \centering \includegraphics[width=0.95\textwidth]{f_1_07}
    \caption{MALDI-ToF spectrum of \emph{p}FF-CE with computed extents of
incorporation based on peak intensities.} \label{fig:MALDI_spectrum_CE} \end{figure}
% --------------------------
\begin{figure}[h!] \centering \includegraphics[width=0.95\textwidth]{f_1_08}
    \caption{MALDI-ToF spectrum of \emph{p}FF-ECE with computed extents of
incorporation based on peak intensities.} \label{fig:MALDI_spectrum_ECE} \end{figure}
% --------------------------
%Amino Acid Analysis Results
Amino acid analysis was calculated based on the internal standardization of
three conserved amino acids - Val, Leu, His - indicated in Table
\ref{tab:aaa_analysis}. The average values are reported in
Table \ref{tab:incorporation_numbers}, with incorporation levels ranging from
\SIrange{76}{88}{\percent}, suggestive of lesser extents of incorporation as
compared to the MALDI-ToF results. 
% --------------------------
%START_TABLE
%AAA Analysis TABLE
\begin{table}[h!]
    \centering
\begin{tabular}{ clllll }
  \hline
  &
  \multicolumn{3}{l}{\emph{p}FF-EC Sample} &
  \multicolumn{2}{l}{Expected Composition} \\
  \hline
  Int Std Residue & nmole Phe & nmole Std & Ratio Phe:Std & Ratio Phe:Std& \emph{p}FF \\
  Leu & 0.1172 & 5.1090 & 0.0229 & 0.1200 & \SI{80.88}{\percent} \\
  Val & 0.1172 & 1.1381 & 0.1030 & 0.5455 & \SI{81.12}{\percent} \\
  His & 0.1172 & 0.6173 & 0.1899 & 1.0000 & \SI{81.01}{\percent} \\
  \hline
  &
  \multicolumn{3}{l}{\emph{p}FF-CE Sample} &
  \multicolumn{2}{l}{Expected Composition} \\
  \hline
  Int Std Residue & nmole Phe & nmole Std & Ratio Phe:Std & Ratio Phe:Std& \emph{p}FF \\
  Leu & 0.1531 & 5.4115 & 0.0283 & 0.1200 & \SI{76.42}{\percent} \\
  Val & 0.1531 & 1.2132 & 0.1262 & 0.5455 & \SI{76.86}{\percent} \\
  His & 0.1531 & 0.6651 & 0.2302 & 1.0000 & \SI{76.98}{\percent} \\
  \hline
  &
  \multicolumn{3}{l}{\emph{p}FF-ECE Sample} &
  \multicolumn{2}{l}{Expected Composition} \\
  \hline
  Int Std Residue & nmole Phe & nmole Std & Ratio Phe:Std & Ratio Phe:Std& \emph{p}FF \\
  Leu & 0.1477 & 11.1488 & 0.0132 & 0.1146 & \SI{88.44}{\percent} \\
  Val & 0.1477 & 1.3925 & 0.1061 & 0.9167 & \SI{88.43}{\percent} \\
  His & 0.1477 & 0.6055 & 0.2439 & 1.8333 & \SI{86.69}{\percent} \\
  \hline
\end{tabular}
\caption{Tabulation of calculations to derive levels of incorporation from amino
acid analysis data}
\label{tab:aaa_analysis}
\end{table}
%END_TABLE
% --------------------------
%Cite comparative incorporation results
The discrepancy between AAA and MALDI-ToF MS estimations for incorporation,
\SIrange[range-phrase=--]{6}{11}{\percent}, are slightly higher than that
observed previously for the incorporation of \emph{p}FF into tGCN5.\cite{Voloshchuk2009}
% --------------------------
%Implications of incomplete incorporation

\subsection{Secondary Structural Anaylsis}
CD spectroscopic wavelength scans were obtained for all three fluorinated
proteins and compared against previously collected spectra for wild-type
variants of the protein polymers, shown in Figure
\ref{fig:CD_temp_wl}.\cite{Haghpanah2009} 

Figure \ref{fig:CD_temp_wl}A suggests the \emph{p}FF-EC is randomly structured
at low temperatures, transitioning to an ${\alpha+\beta}$-rich conformation  at
higher temperatures. By contrast, \emph{p}FF-CE revealed a more structured
${\alpha+\beta}$ conformation at temperatures less than \SI{30}{\celsius}. Upon
heating from \SIrange{30}{50}{\celsius}, \emph{p}FF-CE adopted a predominantly
${\beta}$-rich structure indicated by a single minimum present above
\SI{220}{\nm}. \emph{p}FF-ECE revealed a highly cooperative transition from a
randomly structured protein to a ${\beta}$-rich protein (indicated by a single
minimum at \SI{30}{\celsius}), eventually losing all evidence of structure. The
secondary structures depend on the orientation of the block segments and the
number of the block segments, consistent with the behavior of the wild-type 
proteins.\cite{Haghpanah2009} While both fluorinated diblocks appeared to
exhibit minimal deviation in structure relative to their wild-type analogues,
\emph{p}FF-ECE only resembled its wild-type counterpart at lower temperatures and
conformed to a ${\beta}$-rich structure at higher temperatures. The wild-type
ECE isothermal spectra demonstrated an isodichroic point, conventionally indicative of a
two-state transition, a feature clearly absent in the \emph{p}FF-ECE spectra
collection.

In addition, another noteworthy feature of the isothermal CD spectra
of \emph{p}FF-ECE is the significant loss of signal, concurrent with the
transition to ${\beta}$ conformation. While all block copolymers, both fluorinated
and wild-type, formed visible microscopic particles upon heating, only the
\emph{p}FF-ECE demonstrated a significant loss of signal, suggesting that
\emph{p}FF-ECE recruits more of the soluble protein population into the
collection of larger insoluble particles, relative to the other constructs.
% --------------------------
\begin{figure}[h!] \centering \includegraphics[width=\textwidth]{f_1_12}
    \caption[CD wavelength spectra collected as a function of temperature for
        (a)wt-EC, (b)wt-CE, (c)wt-ECE, (d)\emph{p}FF-EC, (e)\emph{p}FF-CE,
        (f)\emph{p}FF-ECE, indicating the secondary structural changes that
        accompany the thermoresponsiveness of the proteins as well as the
        incorporation of \iupac{\para-fluorophenylalanine}.  Samples were
        prepared to \SI{4}{\micro\moLar} in phosphate buffer initially at
        \SI{4}{\celsius}, establishing congruent preparation protocols to
        previously committed studies on the wild-type protein. Units of CD
    signal expressed as \si{\degtext\cm\squared\per\deci\mol}.]

{CD wavelength spectra collected as a function of temperature for (a)wt-EC,
    (b)wt-CE, (c)wt-ECE, (d)\emph{p}FF-EC, (e)\emph{p}FF-CE, (f)\emph{p}FF-ECE,
    indicating the secondary structural changes that accompany the
    thermoresponsiveness of the proteins as well as the incorporation of
    \iupac{\para-fluorophenylalanine}.  Samples were prepared to
    \SI{4}{\micro\moLar} in phosphate buffer initially at \SI{4}{\celsius},
    establishing congruent preparation protocols to previously committed studies
    on the wild-type protein.\cite{Haghpanah2009} Units of CD signal expressed
    as \si{\degtext\cm\squared\per\deci\mol}.}
    
    \label{fig:CD_temp_wl}
\end{figure}
% --------------------------
%Reserve this section for expositing on the computational derivation of
%structures and more on the reference set used and the rationale for choosing
%said reference set.

The spectral decompositions derived using the CDSSTR implementation for all
three block copolymers, shown in Figure \ref{fig:CD_computation}, suggest an
overall high fractional content consisting of random or unordered structure,
which is consistent with the general disordered nature of elastin-like peptides
at temperatures below their LCST.  While the dominance of unordered composition
tends to persist across the entire temperature ranges explored for all three
polymers, the other secondary structural features, i.e. beta, turn, and helical
content, do obey transitional trends. For \emph{p}FF-EC, the ${\beta}$ content
increases with temperature, consistent with the visual interpretation of the
dataset, which displays a more intense minimum at 208 nm at lower temperatures,
and a more intense minimum at 222 nm at higher temperatures. The diminishing
helical content of \emph{p}FF-CE is consistent with the migration of the
isothermal spectra from a signature defined by dual minima to one of a single
minimum at ca. 220 nm. A similar disappearance of helical content is also
observed for the \emph{p}FF-ECE construct, which experiences a precipitous drop
in helical content at \SI{30}{\celsius}. This cooperative transition of both
signal intensity and structural content is revisited at the supramolecular scale
in the Section \ref{sec:lcst}.

% --------------------------
\begin{figure}[h!]
    \centering
    \begin{subfigure}[b]{0.32\textwidth}
        \centering
        \includegraphics[scale=0.65]{f_1_13_1}
        \caption{\emph{p}FF-EC}
        \label{fig:CDcomp_ECpFF}
    \end{subfigure}
    \begin{subfigure}[b]{0.32\textwidth}
        \centering
        \includegraphics[scale=0.65]{f_1_13_2}
        \caption{\emph{p}FF-CE}
        \label{fig:CDcomp_CEpFF}
    \end{subfigure}
    \begin{subfigure}[b]{0.32\textwidth}
        \centering
        \includegraphics[scale=0.65]{f_1_13_3}
        \caption{\emph{p}FF-ECE}
        \label{fig:CDcomp_ECEpFF}
    \end{subfigure}
    \caption{Secondary structure analysis of protein polymers, performed using
    the CDSSTR method with protein reference set SDP48.}\label{fig:CD_computation}
\end{figure}
% --------------------------
\subsection{Thermoresponsive Supramolecular Assembly}
\label{sec:lcst}
% --------------------------
\begin{figure}[h!]
    \centering
    \begin{subfigure}[b]{0.32\textwidth}
        \includegraphics[width=\textwidth]{f_1_14_1}
        \caption{}
        \label{fig:LCST_ECpFF}
    \end{subfigure}
    \begin{subfigure}[b]{0.32\textwidth}
        \includegraphics[width=\textwidth]{f_1_14_2}
        \caption{}
        \label{fig:LCST_CEpFF}
    \end{subfigure}
    \begin{subfigure}[b]{0.32\textwidth}
        \includegraphics[width=\textwidth]{f_1_14_3}
        \caption{}
        \label{fig:LCST_ECEpFF}
    \end{subfigure}

    \begin{subfigure}[b]{0.32\textwidth}
        \includegraphics[width=\textwidth]{f_1_14_4}
        \caption{}
        \label{fig:LCST_deriv_ECpFF}
    \end{subfigure}
    \begin{subfigure}[b]{0.32\textwidth}
        \includegraphics[width=\textwidth]{f_1_14_5}
        \caption{}
        \label{fig:LCST_deriv_CEpFF}
    \end{subfigure}
    \begin{subfigure}[b]{0.32\textwidth}
        \includegraphics[width=\textwidth]{f_1_14_6}
        \caption{}
        \label{fig:LCST_deriv_ECEpFF}
    \end{subfigure}
    \caption{(a-c) Turbidity observations for EC, CE, and ECE, respectively,
        measured by way of absorbance at \SI{350}{\nm}, from which the bulk
        meso-/macroscale assembly event is concluded. Black lines
        correspond to wild-type data; red lines correspond to \emph{p}FF data. (d-f)
    Computed first derivative of turbidity profiles in a-c.}
    \label{fig:lcst}
\end{figure}
% --------------------------
The effect of temperature on the supramolecular assembly of the proteins was
examined via absorption at \SI{350}{\nm}. The inverse temperature transition
(${T_t}$) was determined via temperature-dependent absorption, shown in Figure
\ref{fig:lcst}, consistent with conventional techniques in studying coacervation
phenomena of elastin-like peptides.\cite{Urry1985} While this technique is more
commonly used to quantify the turbidity of solutions of coacervates, previous
work on these block copolymers used the more resolved techniques of dynamic and
static light scattering in order to obtain size distribution profiles for each
of the proteins during their thermoresponsive transitions. However, this work
details the first application of turbidity measurements to the
wild-type, in addition to the \emph{p}FF-incorporated variants of block copolymers.

In comparison to previously reported results, obtained via dynamic light
scattering, turbidity measurements provided evidence of thermoresponsive
transitions occurring for wild-type proteins in the same range (ca.
\SI{25}{\celsius}), with the onset of transitions occurring close to
\SI{20}{\celsius} for all wild-type proteins (see black curves in Figure 
\ref{fig:lcst}). However, as is evident from turbidity, the process continues
to intensify with respect to absorption, in contrast to the previously obtained
DLS results, which indicated single particle size distributions (for the EC and
ECE proteins) above \SI{20}{\celsius} with no further information regarding the
cooperativity of the transition.

The ${T_t}$ values for all fluorinated block copolymers were different, affirming
the significance of block orientation and number, consistent with the primary
conclusions drafted previously for the wild-type
variants.\cite{Haghpanah2010,Haghpanah2009} The ${T_t}$ of \emph{p}FF-EC was determined
to be \SI{25}{\celsius}, while \emph{p}FF-CE and \emph{p}FF-ECE exhibited a ${T_t}$ at
\SI{28}{\celsius} and \SI{32}{\celsius}, respectively. In comparison to the
wild-type proteins, fluorination appeared to depress the ${T_t}$ of the CE
protein from \SIrange{36}{28}{\celsius}, consistent with the primary hypothesis
regarding the affects of fluorine substitution within the elastin-like peptide
guest residue.

While fluorination did not appear to affect the ${T_t}$ of EC or ECE proteins,
an enhanced cooperativity was observed in the transition for both CE and ECE.
For the CE and ECE constructs, there were also marked increases in the intensity
of the first derivative signal during the transition point, which is an artefact
of the enhanced cooperativity of the transition.  Charts d-f in Figure
\ref{fig:lcst} indicate the first derivative charts of turbidity profiles shown
in charts a-c of the same figure. While Urry and others have defined the
transition temperature associated with lower critical solution temperature
phenomena with the midpoint of the transition temperature
range,\cite{Urry1993,Nuhn2008} the technique of examining the curves' first
derivative was rather employed, which has been championed more recently for
evaluating coacervation
events.\cite{Furgeson2006,Liu2010b,Mackay2010a,Meyer2004} The maxima of each of
the first derivative curves were computed and corresponding temperatures
reported as ${T_t}$ points.  An exception to this approach of analysis was taken
in the case of wt-ECE. While an inflection point in the heating curve exists for
the wt-ECE (see Figure \ref{fig:lcst}f, black curve), the overall curve adheres
to a broader linear profile, which justified a linear interpolation of the curve
between its upper and lower bound plateau regions and determination of the
midpoint of the resultant line. This yielded a transition temperature of
\SI{33}{\celsius} for the wt-ECE.

\begin{figure}[h!] \centering \includegraphics[width=0.95\textwidth]{f_1_22}
    \caption{Electron micrographs of (a) \emph{p}FF-EC, (b) \emph{p}FF-CE, and
    (c) \emph{p}FF-ECE at \SI{4}{\celsius} imaged at ${110000 \times}$ and
    ${19500 \times}$ (inset) magnification. Scale bars represent \SI{200}{\nm}.
    Insets show samples incubated at \SI{40}{\celsius}, with scale bars
    representing \SI{1}{\um}. All proteins prepared to \SI{0.4}{\mg\per\mL}, or
    roughly \SIrange{10}{20}{\micro\moLar}. Micrographs are representative of
    the overall observed morphology of proteins, showing nanometer-sized
    particles that flocculate to larger amorphous assemblies (inset) upon heating.}
    \label{fig:block_EM_pFF}
\end{figure}

Supramolecular assembly of protein constructs was qualitatively confirmed via
transmission electron microscopy studies below and just above their ${T_t}$,
using samples pre-incubated at \SI{4}{\celsius} and \SI{40}{\celsius}.  A panel
of representative micrographs are shown in Figure \ref{fig:block_EM_pFF} for all
three protein samples. Sample concentrations for this experiment were several
fold higher in micromolar concentration than the concentrations prepared for
samples analyzed using CD and UV/Vis spectroscopy.  This allowed for sufficient
density coverage of the sample grid for adequate imagining results; lower
concentrations did not yield sufficient coverage of sample onto the grid,
hindering a qualitative inspection of the sample morphologies.

At these slightly higher concentrations, several key observations of the block
copolymers can be made. First, the \emph{p}FF- proteins, exhibit an amorphous
particle structure on the order of \SIrange{10}{70}{\nm}. At
\SI{4}{\celsius} these particles tend to flocculate to form larger amorphous
assemblies, evident in all micrographs collected at ${19500 \times}$
magnification. Second, when incubated above their ${T_t}$, as it was assessed by
UV/Vis turbidity measurements (see above), micrographs of the samples provide
evidence of larger, micron-scale aggregates. This was particularly noted for
\emph{p}FF-CE (Figure \ref{fig:block_EM_pFF}b) and \emph{p}FF-ECE (Figure
\ref{fig:block_EM_pFF}c). The large population of micron-sized aggregates for
these two samples may explain the significant increase in absorbance intensity
beyond \SI{1.0}{\AU} observed in the corresponding turbidity data shown in Figures
\ref{fig:LCST_CEpFF} and \ref{fig:LCST_ECEpFF}, for \emph{p}FF-CE and
\emph{p}FF-ECE, respectively.

\subsection{Reversibility of Assembly}
\label{sec:reversibility}
% --------------------------
\begin{figure}[h!] \centering \includegraphics[width=0.9\textwidth]{f_1_20}
    \caption[FTIR spectra collected for poly(AVGVP) and poly(GVGVP) vs
        temperature: (a) CH3 a. stretching, (b) lower-frequency part of amide I,
        (c) amide II of poly(AVGVP), (d) CH3 as. stretching, (e) lower-frequency
        part of amide I, (f) amide II of poly(GVGVP)]{FTIR spectra collected for
            poly(AVGVP) and poly(GVGVP) vs temperature: (a) CH3 a. stretching,
            (b) lower-frequency part of amide I, (c) amide II of poly(AVGVP),
            (d) CH3 as. stretching, (e) lower-frequency part of amide I, (f)
            amide II of poly(GVGVP)\cite{Schmidt2005}}\label{fig:lcst_cabello}
        \end{figure}
% --------------------------
Cooling curves were collected directly following the collection of heated-sample
absorption spectra at \SI{320}{\nm}, yielding the dashed curves displayed in
Figure \ref{fig:lcst}. As is evident from these results, the majority of the
constructs were not reversible. In some cases, such as wt-CE and \emph{p}FF-ECE,
there are slight recoveries in the transmittance of the samples. Their
respective counterparts in the study demonstrated negligible recoveries in
transmittance, amounting to less than \SI{10}{\percent} of the total change in
absorbance during the heating process. The absorption curves for both EC-based
proteins demonstrated supramolecular assemblies that continued to transition
beyond the application of additional heat to the sample.

In the case of wt-EC, the sample undergoes a transition that is followed by a
dramatic loss in signal, suggesting that the coacervate assemblies have
agglomerated and settled out of solution as well as out of the light path
(despite the continuous mixing of the sample during the experiment). Thus, while
the curve corresponding to wt-EC (Figure \ref{fig:lcst}A, dotted black) may
appear as a recovery of its initial state, it is more likely a settling of the
protein in the form of large macroscopic particles. This was confirmed by visual
inspection of the samples. For \emph{p}FF-EC, the evolution of a more turbid
solution appears to continue despite the cooling of the sample.

This manner of hysteresis has been observed by other groups working on
elastin-like peptides, with evidence collected as early as 1972, documented in
the work by Walton on partially-hydrolized elastin derived from extracted bovine
\latin{ligamentum nuchae}.\cite{Jamieson1972a} The partial irreversibility of
the formation of aggregate species was concluded from an inability to regain an
original broad spectrum indicative of monomeric or less aggregated species from
light scattering experiments. It was speculated by the authors that the
hysteresis was driven by the electrostatic charge fluctuation mechanism proposed
by Kirkwood and Shumaker\cite{Kirkwood1952} for protein aggregation near the
corresponding isoelectric point. More recently, evidence of hysteresis in
thermally-induced assemblies of elastin-like peptides has been
collected.\cite{Osborne2008,Herrero-Vanrell2005} This has been followed up with
biophysical studies of the hysteresis phenomenon using FTIR, Raman spectroscopy,
and \textsuperscript{2}H NMR.\cite{Schmidt2005,Ma2012b} Rodriguez-Cabello
produced FTIR spectra for two model elastin-like peptides, AVGVP and GVGVP,
demonstrating a significant hysteresis in the thermal transitions of the spectra
corresponding to the \ch{CH3}, amide I, and amide II regions (see Figure
\ref{fig:lcst_cabello}).  Rodriguez-Cabello puts forth that certain side residues
can better expose the peptide backbone to water during the changes in structural
folding in response to temperature, allowing water to act as a softening agent.
For the case of poly(GVGVP), a relatively substantial amount of water molecules
remain between the polymer chains, thereby facilitating the smooth reverse
separation during the backward cooling of the system. Based of the FTIR data
presented in their studies, this is not the case for poly(AVGVP), for which a
majority of the amide groups are bound directly together, leaving few water
molecules in between as well as blocking the penetration of water upon cooling.
It may be the case that the protein block copolymers EC, CE, and ECE, experience
a hysteresis via a similar biophysical phenomenon. That is, the Phe residue that
is located in the similarly matched motif of poly(GFGVP) may be blocking the
repenetration of water in the polypeptide backbone upon cooling. 
% --------------------------
% Also discuss the and reference the paper on mesophase separation.

\subsection{Mechanical Behavior}

Bulk rheological of the protein polymers were evaluated by microrheology, shown
in Figure \ref{fig:pFF_rheology}. Previous work by both Tirrell and Chilkoti
have demonstrated the concentration-dependence of elastin-like peptide
self-assembly.\cite{Meyer2004,Yamaoka2003} Indeed, wt-EC was predominantly
viscous at concentrations below \SI{2.5}{\mg\per\mL} and transitioned to an
elastic network as the concentration was prepared to \SI{10}{\mg\per\mL}, shown
in Figure \ref{fig:wt_rheology}.\cite{Haghpanah2010} This behavior, however was
not observed for wt-CE and wt-ECE, as they existed as predominantly viscous or
viscoelastic suspensions up to \SI{10}{\mg\per\mL}, respectively.
% --------------------------
\begin{figure}[h!] \centering \includegraphics[width=0.95\textwidth]{f_1_15}
    \caption{Microrheological plots of viscous and elastic moduli, derived from
    tracked particle movements within protein solutions. (A,D) \emph{p}FF-EC, (B,E)
    \emph{p}FF-CE, (C,F) \emph{p}FF-ECE, prepared to \SI{1.25}{\mg\per\mL} (A-C) and
    \SI{2.5}{\mg\per\mL} (D-F). Charts display storage (${G'}$, filled markers) and loss
    (${G''}$, empty markers) moduli at \SI{22}{\celsius} (black) and \SI{42}{\celsius}
(red).} \label{fig:pFF_rheology} \end{figure}
% --------------------------

% --------------------------
\begin{figure}[h!] \centering \includegraphics[width=0.95\textwidth]{f_1_16}
    \caption{Microrheological plots of viscous and elastic moduli, derived from
    tracked particle movements within protein solutions. (A,D) wt-EC, (B,E)
    wt-CE, (C,F) wt-ECE, prepared to \SI{2.50}{\mg\per\mL} (A-C) and
    \SI{10}{\mg\per\mL} (D-F). Charts display storage (${G'}$,filled markers) and loss
    (${G''}$, empty markers) moduli at \SI{22}{\celsius} (black) and \SI{42}{\celsius}
(red).} \label{fig:wt_rheology} \end{figure}
% --------------------------
Upon fluorination, more depressed concentration-dependent percolation points for
the protein polymers were observed. While wt-EC was predominantly viscous at
\SI{1.25}{\mg\per\mL}, \emph{p}FF-EC was viscoelastic at \SI{22}{\celsius} at the same
concentration and both demonstrating more elastic character at \SI{42}{\celsius}
at \SI{2.5}{\mg\per\mL} (Figure \ref{fig:pFF_rheology}A,D
\ref{fig:wt_rheology}A,D). Both the wt-CE and \emph{p}FF-CE exhibited
viscous/viscoelastic character at \SI{1.25}{\mg\per\mL} and \SI{2.5}{\mg\per\mL}
at \SI{22}{\celsius}, however, at \SI{2.5}{\mg\per\mL} and \SI{42}{\celsius},
\emph{p}FF-CE demonstrated a dominant elastic modulus compared to the viscoelastic
character of the wt-CE (Figure \ref{fig:pFF_rheology}B,E and Figure
\ref{fig:wt_rheology}B,E). Two types of modifications to the rheology behavior
are observed from the EC and CE sets of constructs. For \emph{p}FF-EC, a transition
shift in the required concentration to yield viscoelastic behavior is
demonstrated, whereas for \emph{p}FF-CE, a rheological shift to elastic behavior was
observed, which was not obtainable, even at concentrations up to
\SI{10}{\mg\per\mL} for wt-CE.\cite{Haghpanah2010}

Similar to both diblocks, \emph{p}FF-ECE underwent a transition and rheological shift.
\emph{p}FF-ECE demonstrated a transition shift from viscous to viscoelastic fluid at
\SI{1.25}{\mg\per\mL} (Figure \ref{fig:pFF_rheology}C and Figure
\ref{fig:wt_rheology}C). Most significantly, \emph{p}FF-ECE at
\SI{1.25}{\mg\per\mL} exhibited elastic network formation at
\SI{42}{\celsius}, while the wt-ECE was completely viscous under the same
concentration and temperature (Figure \ref{fig:pFF_rheology}C and Figure
\ref{fig:wt_rheology}C).

The suppression of elastic behavior of \emph{p}FF-ECE at \SI{42}{\celsius} for the
\SI{2.5}{\mg\per\mL} preparation compared to the \SI{1.25}{\mg\per\mL}
preparation is similar to the gel formation of fibrin clotting networks, which
form ``fine'' transparent networks that demonstrated more elastic behavior
compared to that of ``coarse'' opaque networks.\cite{Clark1987} In general, the
incorporation of \emph{p}FF was observed to facilitate percolation when compared to
their wild-type counterparts. Moreover, fluorination yielded robust elastic
network formation for all three protein polymers at elevated temperatures. 

\section{Discussion}

While the secondary structure appears conserved with respect to diblock
variants, the supramolecular behavior appears to have been altered, suggesting
that even modest effects to the intermolecular interactions can dramatically
affect the self-assembly of macromolecules.\cite{URRY1974} The change in
conformation that accompanies the sample heating corresponds to the
self-assembly that is taking place. For all three protein fluoropolymers, the
supramolecular assemblies as determined by turbidity measurements are dictated
by the transitions from either unstructured to structured as in the case of
\emph{p}FF-EC and \emph{p}FF-ECE, or from ${\alpha+\beta}$-rich to ${\beta}$-rich conformation
as demonstrated by \emph{p}FF-CE. As the proteins assume more ${\beta}$ conformation
due to the elastin composition within the constructs, they undergo self-assembly
into particle aggregates, as also shown in electron micrographs. Indeed, such correlations between CD spectrograms and
turbidimetry profiles have been presented by groups studying elastin-like
peptides; the differential light scattering of polarized light effectively
converts a CD signature to a damped optical rotatory dispersion curve. While
this technique has raised the notion that changes in CD spectra are in fact
losses in signal due to the scattering of incident light upon self-assembly into
large particulates, which is particularly suspected of \emph{p}FF-ECE CD
measurements, the subpopulation that is still in solution displays a marked
transformation in structure.

Although fluorination does not affect the secondary structure or ${T_t}$ of
\emph{p}FF-EC, there is an effect on the ${T_t}$ of \emph{p}FF-CE even though the
temperature-dependent conformation is conserved. The downward shift in the
${T_t}$ of \emph{p}FF-CE can thus be attributed to the promotion of supramolecular
assemblies upon heating, suggested by microrheology, as opposed to alterations
to the evolution of secondary structure with respect to temperature.

In the case of \emph{p}FF-ECE, the secondary structural change correlates with the more
cooperative transition and bulk mechanical response at lower concentration. It
is posited that the pronounced effects on supramolecular assembly upon
fluorinating CE and ECE are due to a synergistic enhancement of the effect of
\emph{p}FF on the hydrophobic collapse of the E domain. Distinct to the CE and ECE is
the dominant ${\beta}$-content as heat is added to the system, in contrast to
the EC protein. The overall results suggest that the hydrophobic collapse and
subsequent ${\beta}$-turn formation of the E domain is occurring and dominating
the compound structure of the block proteins. As part of the dominant structure
for the protein, the \emph{p}FF residues, which are biased toward the E domain, tend to
effect overall changes in the self-assembly process, observed via turbidity and
microrheology experiments.

This aspect of a block system is especially significant when designing block
polymer architectures that undergo biased modification such that the resultant
effects may depend on the dominating structure of the overall polymer, a similar
example of which was recently documented for allosteric actuation of
calmodulin-elastin fusions by Kim and Chilkoti.\cite{Kim2008a} In this example,
a calmodulin domain fused to an elastic-like peptide was able to trigger the
structured collapse of the elastin-like peptide domain upon binding calcium, by
way of switching conformation upon the binding event. Alteration in the tertiary
structure of an appendage domain, such as calmodulin, was able to bias
conformational stability of the elastin domain.

These effects are sometimes more subtle, taking the form of differences in
biosynthetic expression levels, as in the case of the protein fusions developed
by Christensen.\cite{Christensen2009} To this end, limited explanation can be
provided as to the molecular mechanism associated with the effects of fusion
order, with only postulates presented regarding translational dependence of the
protein on the fusion order or easier intracellular degradation of the protein.

These and the plethora of other engineered applications of fusion proteins,
specifically designed with elastin-like peptides as integral domains, underscore
the importance of understanding fusion protein design controls. The simple
aspect of fusion order, as was emphasized in 2009 and 2010 by Haghpanah and
Yuvienco,\cite{Haghpanah2009,Haghpanah2010} should not be overlooked in the
design of these proteins. The work herein demonstrates that this precedent trend
of orientation-dependence is conserved through the process of non-natural amino
acid incorporation.

Fluorination alters the mechanical behavior of all these particular protein
polymers with respect to either concentration or temperature responsiveness.
Specifically, the results demonstrated more elastic character upon incorporation
of \emph{p}FF in all constructs. The data suggests that fluorination promotes
supramolecular association facilitating percolation and elastic network
formation.\cite{Veerman2006,Safran1985,VanderLinden2001} Indeed fluorination of
synthetic polymers has been shown to modulate supramolecular assemblies,
confirming our results with protein
polymers.\cite{Percec2005,Krafft1993,Krafft1994,Krafft2001} While the mechanical
properties of this collection of fusion proteins, particularly the magnitude of
the elastic and viscous moduli, are orders of magnitude less than hydrogel
systems commonly seen in biomedical applications, other systems are often
characterized at higher concentrations and undergo chemical
cross-linking.\cite{Krishna2010,Yan2010a,Rammensee2008,Breedveld2004,Schneider2002}
In fact, the elastic moduli of the constructs presented above are on the same
order of magnitude as \iupac{poly(\L-lysine HBr)-\emph{block}-(\L-leucine)}
polymers developed and characterized by Pine \latin{et al.} and
${\beta}$-hairpin peptides by Pochan \latin{et al.}, demonstrating
\SIrange[scientific-notation=true,retain-unity-mantissa=false]{1e-2}{1e-1}{\pascal}
across
\SIrange[scientific-notation=true,retain-unity-mantissa=false]{1e1}{1e2}{\radian\per\s}.
\cite{Breedveld2004,Yucel2008} We further posit that cross-linking strategies
could be applied to our current system to positively offset the viscoelastic
moduli by orders of magnitude similar to the effects reported by Bausch
\latin{et al.} for recombinantly produced spider
silk.\cite{Rammensee2008,Breedveld2004,Yucel2008}

\subsection{Predictive trends for incorporating fluorine into hydrophobically
collapsible proteins}
\label{sec:predictive_trends}
While there have been numerous studies of the effects of fluorinated amino acids
on altering the stability and structure of proteins, both \latin{de novo} and
natural,\cite{Megeed2002,Meyer2002,Wright2002,Simnick2007,Baker2011,URRY1974,Urry1985}
this work demonstrates that the integration of fluorinated amino acids can also
modify bulk material properties relevant to both the thermoresponsive behavior
of elastin-based fusions and rheological properties of soft gel materials for
biomedical applications. Indeed, the most significant and applicable alterations
to the constructs upon fluorination are manifested in the enablement of unrealized
rheological regimes, as in the case of \emph{p}FF-CE and \emph{p}FF-ECE (Figure
\ref{fig:pFF_rheology}C and E, respectively), and in the shifting of critical
concentrations, as in the case of \emph{p}FF-EC and \emph{p}FF-ECE (Figure
\ref{fig:pFF_rheology}A and E, respectively). Furthermore, these properties of
\emph{p}FF-ECE are also accompanied by a more cooperative transition, making it more
sensitive for applications in temperature-actuated targeting and delivery, an
established application of elastin-based protein
polymers.\cite{Simnick2007,Chilkoti2002a} Modifications to the transition window
have been effected in the past by changes in protein concentration and block
orientation.\cite{Yamaoka2003,Meyer2002} This work exemplifies the potential for
fluorinated residues to narrow the transition window but not significantly alter
the ${T_t}$. Furthermore, larger meso-scale assemblies form in the presence of
fluorinated components, owing to the perhaps fluorophilic assembly that may
drive the intermolecular interactions between single macromolecules.

\subsection{Applications in Drug Delivery}
\label{sec:pff_app_drug}

While the irreversible behavior of this collection of protein block copolymers
may frame the VPG(\emph{p}FF)G motif as an unattractive candidate for
applications in protein purification, this irreversibility of the supramolecular
assembly of the block copolymers makes them potentially attractive for certain drug delivery
applications in the realm of hyperthermia therapy as well as small molecule
encapsulation strategies. An early focus of engineered applications of 
elastin-like peptides by Chilkoti and Meyer, pertained to 
using the elastin-like peptide as a fusion domain to enable thermoresponsive
phase separation of recombinant proteins from soluble
impurities.\cite{Meyer1999,Meyer2001} However, several studies since have reported
irreversible or semi-reversible thermoresponsiveness of elastin-based
constructs. This class of inventions includes silk-elastin biopolymers,
polyelectrolyte-conjugated proteins, and elastin-like peptide
variants.\cite{Herrero-Vanrell2005,Megeed2002,Kayitmazer2007} In the context of
hyperthermia therapy, the mesophase separation of protein constructs can
potentially promote the entrainment of recombinant proteins in the extracellular
space of thermally-targeted tissue, as depicted in Figure
\ref{fig:irreversible_delivery}.
% --------------------------
\begin{figure}[h!] \centering \includegraphics[width=0.5\textwidth]{f_1_23}
    \caption[Scheme proposed by Mackay and Chilkoti, conveying an
        irreversible-associating peptide that is retained in depots after the
        tissue cools to body temperature, contacting a subset of cells. This
        exists as one of several strategies for designing anti-tumor
        temperature-sensitive peptides. Soluble drug carriers are shown to flow
        through permeable tumor vasculature during or after locally applied
        hyperthermia. Carriers irreversibly coalesce in the tumor interstitium
        due to the hysteresis of the thermoresponsive assembly.]

{Scheme proposed by Mackay and Chilkoti,\cite{Mackay2008} conveying an
irreversible-associating peptide that is retained in depots after the tissue
cools to body temperature, contacting a subset of cells. This exists as one of
several strategies for designing anti-tumor temperature-sensitive peptides.
Soluble drug carriers are shown to flow through permeable tumor vasculature
during or after locally applied hyperthermia. Carriers irreversibly coalesce in
the tumor interstitium due to the hysteresis of the thermoresponsive assembly.}

\label{fig:irreversible_delivery} \end{figure}
% --------------------------
The larger morphologies of these block copolymers make them potentially attractive
for injection into, and entrainment within, intra-articular
cavities.\cite{Nettles2008,Betre2006a}

The biophysical reasons behind the irreversibility of assembly of these proteins
may be attributed to effects akin to the kinetically driven assembly of
poly(VPAVG) polymers, as studied by Rodriguez-Cabello.\cite{Reguera2003} In the
preceding case, chain unfolding and rehydration processes, typically associated
with the cooling of fully-reversible elastin-like peptides, occur far from
equilibrium conditions for irreversible cases and further indicate a transition
strongly dominated by kinetics.

%Write about the benefits of more cooperative transitions to drug delivery
%applications.
The enhancement in cooperativity of the thermoresponsive transitions upon
fluorination makes them ideal for hyperthermia therapy drug
delivery.\cite{Furgeson2006} Conventional hyperthermia regimens consist of
localized heating of tumorous tissue, either via focused heating modalities or
induction of systemic physiologic temperatures (ca. \SI{42}{\celsius}), for up
to \SI{2}{\hour} or over several cycles.\cite{Chilkoti2002a,Dreher2007a}
The narrow window of temperature range (\SIrange{39}{43}{\celsius}), within which
hyperthermia therapy may be administered without detrimental cell death due to
overheating, compels the need for delivery vehicles that operate within narrow
thermal windows. A high degree of sharpness in the transition of drug carriers
has thus been highlighted in research efforts.\cite{Nakayama2010,Mackay2008} 

While the COMP domain of the constructs was not interrogated as much for its
functional properties as was the elastin domain for its transitional behavior,
it should be noted that the former is highly attractive for drug delivery
applications and has shown to enhance the small molecule absorption properties
of elastin-like peptides.\cite{Haghpanah2010}

\subsection{Applicability of fluorinated proteins to \textsuperscript{19}F
\label{sec:pff_app_mri}
magnetic resonance imagining modalities} 

While the results in Section: \ref{sec:biosynthesis_results} demonstrate that
residue-specific incorporation methods can successfully be applied to the block
copolymers with high efficiency (\textgreater \SI{80}{\percent}), there are
subtle effects on the physicochemical properties of the polymers that are
relevant to the potential of these polymers to be applied as therapeutic
delivery agents equipped with MRI detectable \textsuperscript{19}F. The need to
focus the concentration of \textsuperscript{19}F in the physiological region of
interest is tantamount to the viable technical applicability of the constructs,
and is thereby reliant on their ability to assemble into supramolecular assemblies of
high density.
% --------------------------

The sub-\SI{100}{\nm} diameter size distribution of the protein polymer at
\SI{4}{\celsius}, (see Figure \ref{fig:block_EM_pFF}) as well as the consistent
baseline in turbidity profiles for \emph{p}FF-CE and \emph{p}FF-ECE (see Figure
\ref{fig:lcst}) prior to their temperature-induced transitions makes them ideal
delivery agents in this respect. The slight flocculation of sub-\SI{100}{\nm}
particles at \SI{4}{\celsius}, which results in aggregates with diameters of
\SIrange{100}{500}{\nm} in size makes them more ideal for extravasation into
tumor vasculature. This is in consideration of previous studies regarding the
enhanced permeability and retention (EPR) effect, which probed the dependency of
this effect on the size of macromolecular assemblies. Pore size cutoffs for
tumor vasculature have been reported from these studies to manifest within
\SIrange{400}{600}{\nm} diameters,\cite{Yuan1995} with maximal absorption into
tissues occurring with supramolecular assemblies with diameters of
\SI{100}{\nm}.\cite{Charrois2003} The nano-aggregates formed by \emph{p}FF-CE
and \emph{p}FF-ECE tend to flocculate further to form micron-scale assemblies
(see Figure \ref{fig:block_EM_pFF} insets). The lower absorbance of the
wild-type as well as precedent electron microscopy studies of these variants
(see Figure \ref{fig:block_EM_wt}) suggest that they do not tend to coalesce as
extensively as their fluorinated counterparts. This positions the fluorinated
variants as more ideal candidates for concentrating not only small molecule
payload, but also \textsuperscript{19}F-labeled residues.
% --------------------------

Preliminary \textsuperscript{19}F ssNMR data for a variant of these block
polymers, containing three repeating elastin domains (as opposed to five), is
presented in Section \ref{sec:ss_nmr}. The experiment aimed to elucidate any
thermoresponsive behavior of the protein as a solid mass, when packed into an
NMR rotor, by monitoring the spin-lattice relaxation time, ${T_{1}}$ at
\SI{10}{\celsius} and \SI{35}{\celsius}. While the results were inconclusive
regarding differences in relaxation behavior, the strong signal pertaining to the
\textsuperscript{19}F in the protein demonstrated the applicability of these
constructs as MRI contrast agents. Aside from the various attempts to monitor the
extravasation of liposomal drug carriers via Gd-labeling,\cite{Tagami2011} this
is the first attempt toward the integration of fluorinated amino acids into
elastin-based delivery vehicles, proving the technical feasibility for further
studies, test applications \latin{in vitro} and \latin{in vivo}, and bioprocess
optimization.

\subsection{Bottlenecks of Biosythesis}

While the field of protein engineering has offered numerous examples of
\textsuperscript{19}F-labeled proteins, by way of the incorporation of
non-canonical fluorinated amino acids (as presented in Section
\ref{sec:fluorinated_amino_acids}), it is the vehement opinion of this author
that the costs associated with the synthesis and incorporation of non-natural
amino acids,\cite{Hodgson2004,Ojima1989} particularly when applied to the
screening of several mutants prior to pilot-scale development/application, 
situates the realization of fluorinated protein-based delivery agents as a far
reality.\footnote{At the time of this publication, \iupac{trifluoro-\L-leucine} was
distributed from U.S. manufacturers for
\$\SIrange[range-phrase=--]{300}{500}{\per\gram}.}

Yet another challenge still exists with respect to controlling the concentration
of the proteins such that stock concentrations are not too high so as to depress
the ${T_t}$ of the stock preparation, resulting in a premature irreversible
coacervation event. Furthermore, growth of elastin-like peptide-based
macromolecules with ${T_t}$ points near the operating temperatures of the growth
conditions of bioreactors may result again in premature meso-phase separation.

\subsection{Protein design}

In the realm of synthetic chemistry, there has been a long standing interest in
the physicochemical properties of fluorinated polymers. Self-assembly into
higher-order structures has gained particular focus, in the cases of
semifluorinated dendritic Janus particles and fluorinated amphiphiles, which
affect assemblies on the supramolecular scale in different ways, and
despite the early successes in the incorporation of fluorinated amino acids into
protein polymers, little has been accomplished in the field with respect to
material characterization. Our studies demonstrate that fluorinating biopolymers
cannot only impact the secondary structure and ${T_t}$, but, more importantly,
influence the supramolecular assemblies and mechanical properties. While these
fluorinated protein polymers exist as soft gels, the observed modifications to
the self-assembly and rheological properties from the incorporation of
non-natural amino acids provides a precedence and an opportunity for tuning
protein-based materials. This provides a novel and alternative route for tuning
smart materials that rely on gel mechanics, in the case of applications in
tissue engineering, and thermoresponsive transition, in the case of applications
in drug delivery. 

\subsection{Future work}

Further optimization devoted to the incorporation efficiencies of fluorinated
amino acids into these block copolymers is necessary for further practical
development of the compositions. More uniform incorporation will further promote
consistency in the observable physicochemical properties of the block
copolymers, notwithstanding any inherent stochastic behavior that embodies the
self-assembly processes. This may be carried out by either carrying out the
biosynthetic expression under stricter selective control, or by engineering the
expression system to one more based on orthogonal transcription technologies.

Beyond the improvement in incorporation methods, additional attention should be
devoted to the functional characteristics of COMP, as it exists as part of the
block copolymer. These observations on COMP - specifically its ability to 1)
oligomerize and 2) bind to small molecules - should be carried out at conditions
preceding and following thermo-actuated transitions, as they are measured in the
body of this work. This work may seek, for example, to correlate the thermal
transition of the elastin domain to the incrementation of COMP oligomerization
states, providing insight into the engineered application of elastin-like
peptides as oligomerization chaperones. Assessment of binding properties of the
COMP domain, and/or the other elements of the block copolymers, will further
promote these inventions toward \latin{in vitro} and \latin{in vivo}
applications; the challenge lies with adopting viable small molecule as payload
candidates.

%\printbibliography[heading=subbibliography]

\end{refsection}
