\section*{Preface}
Thank you for participating in the evaluation of my dissertation. The work
documented in this final draft of my dissertation is entitled:
\hspace{0pt}\\
\hspace{0pt}\\
\emph{Modulation of the Physicochemical Properties of Block Protein Copolymers
via Fluorination and the Adaptation of Coiled-coil Proteins for Small Molecule
Delivery Applications}.
\hspace{0pt}\\
\hspace{0pt}\\
This thesis documents two separate but related efforts that have spanned the
last 4 years pertaining to:
   
\begin{enumerate}

    \item The incorporation of a fluorinated non-natural amino acid into a
        protein block co-polymer consisting of elastin-like peptide and
        coiled-coil domains, and the resultant effects on on its physicochemical
        properties, including secondary structure and emergent mechanical
        properties.

    \item The adaptation of the coiled-coil domain of the cartilage oligomeric
        matrix protein to the delivery of a retinoid inverse agonist, a
        therapeutic candidate for the treatment of osteoarthritis

\end{enumerate}

Your previous comments, both on the pre-defense draft of this documents as well
as during my pre-defense presentation, were very much appreciated. Attention was
given to providing more thorough discussion sections and figure descriptions.
The utmost consideration was given to your critiques, and are reflected in this
document.  For your convenience, I have paraphrased and itemized the issues
brought forth in the committee's letter in response to my pre-defense in the
following pages.  The manner in which each issue was addressed in this document
is also delineated. The original letter following my qualifying examination is
also provided.

Thank you very much again, and I sincerely welcome any editorial comments you
may have in the near future prior to final submission, as well as our final
defense meeting on the afternoon of Wednesday, September 10th, 2014 at 3PM in
RH603.

\hspace{0pt}\\
Sincerely,
\hspace{0pt}\\
\hspace{0pt}\\
Carlo Yuvienco

\begin{landscape}
% --------------------------
\renewcommand{\arraystretch}{1.5}
%START_TABLE
\begin{table}[h!]
    \centering
    \begin{tabular}{ p{0.64\textwidth} p{0.64\textwidth} }
    %\begin{tabular}{ p{0.5\textwidth} p{0.5\textwidth} }
    \hline
    \multicolumn{2}{c}{Chapter I Editorial Review} \\
    \hline
    \multicolumn{1}{c}{Critique} &
    \multicolumn{1}{c}{Manner of Address} \\
    \hline
    
    Calculate the thermodynamic values based on UV/Vis melt curve data.
    &

    \\

    Explain the observation of the \emph{p}FF-ECE exhibiting a loss in
    elasticity at higher concentrations with brightfield microscopy data.
    &
    \\

    Provide a path forward as to how solid state experiments can lead to imaging
    using \textsuperscript{19}F MRI.
    &
    \\

    \hline
\end{tabular}
\end{table}
%END_TABLE
% --------------------------
\renewcommand{\arraystretch}{1.5}
%START_TABLE
\begin{table}[h!]
    \centering
    \begin{tabular}{ p{0.64\textwidth} p{0.64\textwidth} }
    %\begin{tabular}{ p{0.5\textwidth} p{0.5\textwidth} }
    \hline
    \multicolumn{2}{c}{Chapter II Editorial Review} \\
    \hline
    \multicolumn{1}{c}{Critique} &
    \multicolumn{1}{c}{Manner of Address} \\
    \hline
    
    Using Beer-Lambert's law, you should be able to calculate the concentration
    of the sample and assess what is really going on with zonal elution
    experiments; it may be that perhaps the protein is sticking to the column or
    surface.
    &
    \\

    The estimate that the \SI{3200}{\Da} represents 8 molecules of BMS493 from
    the calibration curve may not be valid as the error is within
    \SI{10}{\percent}.
    &
    \\

    Endotoxin contamination might be the cause for the catabolic response and
    that is one possibility under investigation; is there a positive and
    negative control that you could devise and test?
    &
    \\

    Are there experiments that you might be able to design to assess the ability
    for COMPcc alone to sequester ATRA leading to OA treatment?
    &
    \\

    Indicate alternative approaches to the Hummer-Dreyer method as presently it
    has not been validated as a suitable approach for determining binding
    affinities for BMS493.
    &
    \\

    \hline
\end{tabular}
\end{table}
%END_TABLE
% --------------------------
\end{landscape}
\renewcommand{\arraystretch}{1}

\includepdf[pages={1,2}]{letter.pdf}
