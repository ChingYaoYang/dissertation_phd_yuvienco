\section*{Preface}
Your final comments regarding my thesis were very much appreciated.  Attention
was given to each point addressed in the committee letter, and the utmost
consideration was given to your critiques, reflected in this document.
For your convenience, I have paraphrased and itemized the issues
brought forth in the committee's letter in response to my defense presentation in the
following pages.  The manner in which each issue was addressed in this document
is also delineated. 

Please notify me at your earliest possible convenience if these changes are
satisfactory. Pending your confirmation, I will require your signature on final
bound copies of this document prior to Friday, September 19, 2014. If you can
let me know an available time on either Thursday, September 18, or Friday,
September 19, during which you can sign the bound copies, I would greatly
appreciate it.

Thank you very much again, and you have my most sincere gratitude for your
participation.

\hspace{0pt}\\
Sincerely,
\hspace{0pt}\\
\hspace{0pt}\\
Carlo Yuvienco

\begin{landscape}
% --------------------------
\renewcommand{\arraystretch}{1.5}
%START_TABLE
\begin{table}[h!]
    \centering
    \begin{tabular}{ p{0.64\textwidth} p{0.64\textwidth} }
    %\begin{tabular}{ p{0.5\textwidth} p{0.5\textwidth} }
    \hline
    \multicolumn{2}{c}{Chapter I Editorial Review} \\
    \hline
    \multicolumn{1}{c}{Critique} &
    \multicolumn{1}{c}{Manner of Address} \\
    \hline
    
    Calculate the thermodynamic values based on UV/Vis melt curve data.
    &
    Thermodynamic values are now calculated based on UV/Vis data (Section
    \ref{sec:thermo_method}) and reported in Sections \ref{sec:thermo_analysis},
    and \ref{sec:thermo_discussion}.
    \\

    Explain the observation of the \emph{p}FF-ECE exhibiting a loss in
    elasticity at higher concentrations with brightfield microscopy data.
    &
    Anomalous microrheological data pertaining to \emph{p}FF-ECE is addressed
    with further discussion (based on defense presentation points) and
    brightfield micrographs of sample groups (Section \ref{sec:ECE_explanation}).
    \\

    Provide a path forward as to how solid state experiments can lead to imaging
    using \textsuperscript{19}F MRI.
    &
    Additional discussion is provided on future potential for
    \textsuperscript{19}F MRI based on solid-state data presented in the defense
    presentation and in Section \ref{sec:ss_nmr}.
    \\

    \hline
\end{tabular}
\end{table}
%END_TABLE
% --------------------------
\renewcommand{\arraystretch}{1.5}
%START_TABLE
\begin{table}[h!]
    \centering
    \begin{tabular}{ p{0.64\textwidth} p{0.64\textwidth} }
    %\begin{tabular}{ p{0.5\textwidth} p{0.5\textwidth} }
    \hline
    \multicolumn{2}{c}{Chapter II Editorial Review} \\
    \hline
    \multicolumn{1}{c}{Critique} &
    \multicolumn{1}{c}{Manner of Address} \\
    \hline
    
    Using Beer-Lambert's law, you should be able to calculate the concentration
    of the sample and assess what is really going on with zonal elution
    experiments; it may be that perhaps the protein is sticking to the column or
    surface.
    &
    Beer-Lambert calculations of protein concentration based on chromatogram
    integrations was attempted, but considered incompatible with the dataset
    based on the limited collection of absorption data at \SI{280}{\nm} and the
    lack of Trp, Tyr, and Cys residues in the sequence. Discussion is however
    added to Section \ref{sec:ze_premature_assembly} that addresses the possible
    explanation of protein adherence to the LC system for the evolution of
    time-course chromatograms. 
    \\

    The estimate that the \SI{3200}{\Da} represents 8 molecules of BMS493 from
    the calibration curve may not be valid as the error is within
    \SI{10}{\percent}.
    &
    Conclusions of the exact molar ratios of binding based on SEC data have been
    re-evaluated to reflect the discussion points regarding measurement error
    during the defense presentation (see Section \ref{sec:discuss_binding}).
    \\

    Endotoxin contamination might be the cause for the catabolic response and
    that is one possibility under investigation; is there a positive and
    negative control that you could devise and test?
    &
    Future work discussion is added to the document (see Section
    \ref{sec:future_work_endotoxin}) that addresses this point.
    \\

    Are there experiments that you might be able to design to assess the ability
    for COMPcc alone to sequester ATRA leading to OA treatment?
    &
    Future work discussion is added to the document (see Section
    \ref{sec:future_work_ATRA_sequestration}) that addresses this point.
    \\

    Indicate alternative approaches to the Hummer-Dreyer method as presently it
    has not been validated as a suitable approach for determining binding
    affinities for BMS493.
    &
    Sedimentation equilibrium is proposed and discussed as a viable alternative
    approach to evaluate binding affinities of oligomeric COMPcc to BMS493 (see
    Section \ref{sec:discuss_binding}).
    \\

    \hline
\end{tabular}
\end{table}
%END_TABLE
% --------------------------
\end{landscape}
\renewcommand{\arraystretch}{1}
