\section*{Preface}
Thank you for participating in the evaluation of my dissertation. The work
documented in this final draft of my dissertation is entitled:
\hspace{0pt}\\
\hspace{0pt}\\
\emph{Modulation of the Physicochemical Properties of Block Protein Copolymers
via Fluorination and the Adaptation of Coiled-coil Proteins for Small Molecule
Delivery Applications}.
\hspace{0pt}\\
\hspace{0pt}\\
This thesis documents two separate but related efforts that have spanned the
last 4 years pertaining to:
   
\begin{enumerate}

    \item The incorporation of a fluorinated non-natural amino acid into a
        protein block co-polymer consisting of elastin-like peptide and
        coiled-coil domains, and the resultant effects on on its physicochemical
        properties, including secondary structure and emergent mechanical
        properties.

    \item The adaptation of the coiled-coil domain of the cartilage oligomeric
        matrix protein to the delivery of a retinoid inverse agonist, a
        therapeutic candidate for the treatment of osteoarthritis

\end{enumerate}

Your previous comments, both on the pre-defense draft of this documents as well
as during my pre-defense presentation, were very much appreciated. Attention was
given to providing more thorough discussion sections and figure descriptions.
The utmost consideration was given to your critiques, and are reflected in this
document.  For your convenience, I have paraphrased and itemized the issues
brought forth in the committee's letter in response to my pre-defense in the
following pages.  The manner in which each issue was addressed in this document
is also delineated. The original letter following my qualifying examination is
also provided.

Thank you very much again, and I sincerely welcome any editorial comments you
may have in the near future prior to final submission, as well as our final
defense meeting on the afternoon of Wednesday, September 10th, 2014 at 3PM in
RH603.

\hspace{0pt}\\
Sincerely,
\hspace{0pt}\\
\hspace{0pt}\\
Carlo Yuvienco

\begin{landscape}
% --------------------------
\renewcommand{\arraystretch}{1.5}
%START_TABLE
\begin{table}[h!]
    \centering
    \begin{tabular}{ p{0.64\textwidth} p{0.64\textwidth} }
    %\begin{tabular}{ p{0.5\textwidth} p{0.5\textwidth} }
    \hline
    \multicolumn{2}{c}{Chapter I Editorial Review} \\
    \hline
    \multicolumn{1}{c}{Critique} &
    \multicolumn{1}{c}{Manner of Address} \\
    \hline
    
    Use existing data to better link the structural changes to supramolecular
    assembly/aggregation. More analysis and discussion is requested on this
    issue.
    &
    Connections are drawn between secondary structural information and
    supramolecular assembly in Section \ref{sec:reversibility}.
    \\

    Provide evidence to support the attribution of the structural differences
    \emph{p}FF-ECE to light scattering.
    &
    UV/Vis data are now reported in terms of absolute absorption, as opposed to
    normalized values (Figure \ref{fig:lcst}). In addition, TEM micrographs
    (Figure \ref{fig:block_EM_pFF}) are presented for the constructs, providing
    additional evidence of meso-scale assemblies that may contribute to light
    scattering and resultant loss in signal for \emph{p}FF-ECE.
    \\

    Clarify the extent reversibility of the temperature-induced transitions of
    the block copolymers.
    &
    Cooling curves are included in Figure \ref{fig:lcst}, indicating irreversible
    transitions for all constructs.
    \\

    Put trends in aggregate size and/or temperature responsiveness in the
    context of molecular mechanisms.
    &
    Allusions to molecular mechanistic models and their bearing on the
    phenomenological behavior of the constructs are presented in Section
    \ref{sec:reversibility}.
    \\

    Discuss extractable generalizations and paradigms that can enable others to
    make predictions.
    &
    Generalizations and predictive trends are now given its own focus in Section
    \ref{sec:predictive_trends}.
    \\

    Evaluate how these resulting fluorinated protein polymers can be used for
    applications.
    &
    An evaluation of the applicability of constructs is presented in Sections
    \ref{sec:pff_app_drug} and \ref{sec:pff_app_mri}.
    \\

    \hline
\end{tabular}
\end{table}
%END_TABLE
% --------------------------
\renewcommand{\arraystretch}{1.5}
%START_TABLE
\begin{table}[h!]
    \centering
    \begin{tabular}{ p{0.64\textwidth} p{0.64\textwidth} }
    %\begin{tabular}{ p{0.5\textwidth} p{0.5\textwidth} }
    \hline
    \multicolumn{2}{c}{Chapter II Editorial Review} \\
    \hline
    \multicolumn{1}{c}{Critique} &
    \multicolumn{1}{c}{Manner of Address} \\
    \hline
    
    Include all data and results relevant to \latin{in vitro} experiments,
    which were referenced / alluded to during the pre-defense presentation.
    &
    See Sections
    \ref{sec:bms493_results}, 
    \ref{sec:endotoxins}, and 
    \ref{sec:mmp13} pertaining to the effects of COMP and COMP+BMS493 on
    the chondrocyte transcriptome. 
    \\

    Explain whether and how BMS493 will be delivered from the protein, citing
    the support for the postulated diffusion model release mechanism.
    &
    The Discussion section is in part devoted to the binding mechanisms of
    BMS493 to COMP, both at the intermolecular scale and the meso-scale (Section
    \ref{sec:discuss_binding}), with additional commentary on the ramifications
    of this model on \latin{in vivo} fate (Section \ref{sec:invivo_fate}).
    \\

    Comment on the fate of the protein delivery agent following release.
    &
    Speculation is presented in Section \ref{sec:invivo_fate} on the
    intra-articular (IA) fate of COMP following the hypothetical release of
    small molecule payload. Analogies are proposed to fates of other
    microspherical delivery agents during IA delivery.
    \\

    Speculate the reasons for protein acting catabolically on articular
    cartilage.
    &
    Speculation as to the reasons underlying the over-expression of catabolic
    markers, seemingly induced by COMP (see Figure \ref{fig:pcr_panel}), is
    presented in Section \ref{sec:comp_catabolism}.
    \\

    Connect the biophysical work on COMP to the cell culture studies.
    &
    An explanation of the inhibition of hypertrophic differentiation of
    chondrocytes when treated with COMP or COMP+BMS493 is presented in Section
    \ref{sec:discuss_hypertrophy} and borrows from the results of the
    biophysical study conducted herein and documented in the precedent
    literature.
    \\

    \hline
\end{tabular}
\end{table}
%END_TABLE
% --------------------------
\end{landscape}
\renewcommand{\arraystretch}{1}

\includepdf[pages={1,2}]{letter.pdf}
